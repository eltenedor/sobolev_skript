\chapter{Some Facts about Lebesgue Spaces \texorpdfstring{$\Ell^p(\Omega)$}{L\textasciicircum p(Omega)}}

Here, we recall some facts about Lebesgue spaces which should be known from previous lectures.

Throughout this lecture, a set $\Omega \subset \R^n$ is called \emph{measurable} if it is measurable w.r.t. the Lebesgue measure on $\R^n$.
Unless otherwise stated, we always assume in this chapter that $\Omega \subset \R^n$ is measurable.

Then $u \colon \Omega \to [-\infty,\infty]$ is measurable \emph{on} $\Omega$ if $\{x \in \Omega \colon u(x) > \alpha\}$ is measurable for any $\alpha \in \R$.

\section{\texorpdfstring{$\Ell^p(\Omega)$}{L\textasciicircum p(Omega)}: Definition and Basic Properties}
\label{sec:lpBasics}
\begin{enumerate}[i)]
  \item If $u,v \colon \Omega \to [-\infty,\infty]$ are measurable on $\Omega$, they are equivalent if $u = v$ a.e. in $\Omega$.
    $[u]$ is the equivalence class of $u$.
    We always identify a function $u$ with its equivalence class.
  \item For $p \in [1,\infty]$, we define the Lebesgue space
    $$
    \Ell^p(\Omega) \coloneqq \{ u \colon \Omega \to [-\infty,\infty] \colon u \text{ measurable }, \|u\|_{\Ell^p(\Omega)} < \infty \},
    $$
    where
    \begin{align*}
      \|u\|_{\Ell^p(\Omega)} &= \left( \int_\Omega |u(x)|^p \d x \right)^{\frac{1}{p}} \text{ if } p \in [1,\infty), \\
        \|u\|_{\Ell^\infty(\Omega)} &= \esssup_{x \in \Omega} |u(x)|.
    \end{align*}
    With the convention from i), $u = 0$ in $\Ell^p(\Omega)$ if $u = 0$ a.e. in $\Omega$.
    If $[u]$ contains a continuous function, we assume that $u$ is chosen to be continuous.
  \item $\Ell^p(\Omega)$ is a Banach space for $p \in [1,\infty]$, i.e. a complete and normed vector space.
  \item $\Ell^p$-convergence and a.e.-convergence: Let $p \in [1,\infty]$, $(u_n)_{n \in \N} \subset \Ell^p(\Omega)$ and $u \in \Ell^p(\Omega)$, such that $u_n \to u$ in $\Ell^p(\Omega)$, i.e. $\|u_n - u\|_{\Ell^p(\Omega)} \to 0$ as $n \to \infty$.
    Then there is a subsequence $(u_{n_k})_{k \in \N}$ and a function $h \in \Ell^p(\Omega)$ such that $u_{n_k}(x) \to u(x)$ a.e. in $\Omega$ as $k \to \infty$ and $|u_{n_k}(x)| \leq h(x)$ a.e. in $\Omega$ for all $k \in \N$.
  \item Minkowski's inequality: Let $1 \leq p \leq \infty$ and $u,v \in \Ell^p(\Omega)$.
    Then
    $$
    \|u + v\|_{\Ell^p(\Omega)} \leq \|u\|_{\Ell^p(\Omega)}+ \|v\|_{\Ell^p(\Omega)}.
    $$
  \item Hölder's inequality: Let $p,q \in [1,\infty]$ with $\frac{1}{p} + \frac{1}{q} = 1$ and $u \in \Ell^p(\Omega), v \in \Ell^q(\Omega)$. 
    Then $uv \in \Ell^1(\Omega)$ and
    $$
    \|uv\|_{\Ell^1(\Omega)} \leq \|u\|_{\Ell^p(\Omega)} \|v\|_{\Ell^q(\Omega)}.
    $$
  \item For $x,y \in \R^n$,
    \begin{align*}
      \|x\|_p &= \left( \sum_{i = 1}^n |x_i|^p  \right)^{\frac{1}{p}} \quad\text{if } p \in [1,\infty), \\
        \|x\|_{\infty} &= \max_{i} |x_i|
    \end{align*}
    the discrete versions of v), vi) are valid:
    \begin{align*}
      \|x + y\|_p &\leq \|x\|_p + \|y\|_p, \\
      |x \cdot y| &\leq \|x\|_p \|y\|_q \quad\text{for } p \in [1, \infty], \frac{1}{p} + \frac{1}{q} =1.
    \end{align*}
  \item General Hölder inequality: Let $p_k \in [1,\infty], \frac{1}{p_1} + \cdots+ \frac{1}{p_m} = 1$, $m \geq 3$, $u_k \in \Ell^{p_k}(\Omega)$, and $k = 1,\dots,m$.
    Then
    $$
    \int_\Omega |u_1 \cdot \ldots \cdot u_m| \d x \leq \prod_{k = 1}^m \|u_k\|_{\Ell^{p_k}(\Omega)}.
    $$
\end{enumerate}

\section{Limit Theorems and Fubini}

\begin{enumerate}[i)]
  \item Monotone convergence (Beppo-Levi): Let $(u_n)_{n \in \N}$ be measurable in $\Omega$, non-negative, and point-wise non-decreasing. Then
    $$
    \int_\Omega \left( \lim_{n \to \infty} u_n(x) \right) \d x = \lim_{n \to \infty} \int_\Omega u_n(x) \d x.
    $$
  \item Fatou's lemma: Let $(u_n)_{n \in \N}$ be measurable in $\Omega$ and non-negative. Then
    $$
    \int_\Omega \left( \liminf_{n \to \infty} u_n(x) \right) \d x 
    \leq \liminf_{n \to \infty} \int_\Omega u_n(x) \d x.
    $$
  \item Dominated convergence (Lebesgue): Let $(u_n)_{n \in \N}$ and $u$ be measurable on $\Omega$ such that $u_n(x) \to u(x)$ as $n \to \infty$ a.e. in $\Omega$ and $|u_n(x)| \leq h(x)$ a.e. in $\Omega$ for all $n \in \N$ and some $h \in \Ell^1(\Omega)$.
    Then $u_n, u \in \Ell^1(\Omega)$ for all $n \in \N$ and $\lim_{n \to \infty} \int_\Omega u_n(x) \d x = \int_\Omega u(x) \d x$.
  \item Fubini's theorem: Let $u = u(x,y)$ be measurable on $\R^{n + m}$ such that at least one of the following integrals exists and is finite:
    \begin{align*}
      I_1 &= \int_{\R^{n + m}} |u(x,y)| \d x \d y, \\
      I_2 &= \int_{\R^m} \left( \int_{\R^n} |u(x,y)| \d x \right) \d y, \\
      I_2 &= \int_{\R^n} \left( \int_{\R^m} |u(x,y)| \d y \right) \d x.
    \end{align*}
    Then $u(\cdot,y) \in \Ell^1(\R^n)$ for a.e. $y \in \R^m$, $\int_{\R^m} u(\cdot,y) \d y \in \Ell^1(\R^n)$, $u(x,\cdot) \in \Ell^1(\R^m)$ for a.e. $x \in \R^n$, $\int_{\R^n} u(x, \cdot) \d x \in \Ell^1(\R^m)$, and $I_1 = I_2 = I_3$.
\end{enumerate}

\section{Dense Subspaces and Mollifiers}
\label{sec:mollifier}

In this section let $\Omega \subset \R^n$ be open.
\begin{enumerate}[i)]
  \item $\CC_0^\infty(\Omega)$ is dense in $\Ell^p(\Omega)$ for any $p \in [1,\infty)$, i.e. for any $u \in \Ell^p(\Omega)$ and $\varepsilon > 0$ there is $\varphi \in \CC_0^\infty(\Omega)$ such that $\|\varphi - u\|_{\Ell^p(\Omega)} < \varepsilon$.
    \item Notation: For $\varepsilon > 0, x \in \R^n$, let 
      \begin{align*}
        \BB_\varepsilon(x) &\coloneqq \{ y \in \R^n \colon |y - x| < \varepsilon \}, \\
        \Omega_\varepsilon &\coloneqq \{ x \in \Omega \colon \dist(x, \partial\Omega) > \varepsilon \}, \quad\text{ and } \\
        \Ell_{\loc}^p(\Omega) &\coloneqq \{ u \colon \Omega \to [-\infty, \infty] \colon u \in \Ell^p(V) \text{ for all } V \Subset \Omega \} \quad\text{for $p \in [1,\infty]$.}
    \end{align*}
  \item Standard mollifier: Let
    $$
    \eta(x) \coloneqq \begin{cases} c \exp\left( \frac{1}{|x|^2 - 1} \right), &|x| < 1, \\ 0, &|x| \geq 1, \end{cases}
    $$
    where $c > 0$ is chosen such that $\int_{\R^n} \eta(x) \d x = 1$.
    Then $\eta \in \CC_0^\infty(\R^n)$ is called \emph{standard mollifier}.
    For $\varepsilon > 0$,
    $ \eta_\varepsilon(x) \coloneqq \frac{1}{\varepsilon^n} \eta \left(\frac{x}{\varepsilon} \right)$, $x \in \R^n$,
    satisfies $\eta_\varepsilon \in \CC_0^\infty(\R^n)$, $\int_{\R^n} \eta_\varepsilon(x) \d x = 1$, and $\supp(\eta_\varepsilon) = \overline{\BB_\varepsilon(0)}$.
  \item For $u \in \Ell^1(\Omega)$, we extend $u$ by $u(x) \coloneqq 0$ for all $x \in \R^n \setminus \Omega$ to $u \in \Ell^1(\R^n)$ and define its \emph{mollification} $u_\varepsilon \coloneqq \eta_\varepsilon \ast u$ for $\varepsilon > 0$, i.e.
    $$
    u_\varepsilon(x) 
    = \int_{\R^n} \eta_\varepsilon(x - y) u(y) \d y
      = \int_{\Omega} \eta_\varepsilon(x - y) u(y) \d y 
      = \int_{\BB_\varepsilon(x) \cap \Omega} \eta_\varepsilon(x - y) u(y) \d y, \quad x \in \R^n.
    $$
\end{enumerate}
The mollification has the following properties:
\begin{thm}
  \label{thm:mollifier}
  Let $u \in \Ell^1(\Omega)$ and $\varepsilon > 0$. Then the following statements hold true:
  \begin{enumerate}[a)]
    \item $u_\varepsilon \in \CC^\infty(\R^n)$, $u_\varepsilon(x) \to u(x)$ for a.e. $x \in \Omega$ as $\varepsilon \downarrow 0$.
    \item If $\supp(u) \Subset \Omega$, then $u_\varepsilon \in \CC_0^\infty(\Omega)$ for small enough $\varepsilon$. 
    \item If $u \in \CC^0(\Omega)$, $V \Subset \Omega$, then $u_\varepsilon \to u$ uniformly in $V$ as $\varepsilon \downarrow 0$.
    \item If $u \in \Ell^p(\Omega)$ for some $p \in [1,\infty)$, then $u_\varepsilon \in \Ell^p(\Omega)$, $\|u_\varepsilon\|_{\Ell^p(\Omega)} \leq \|u\|_{\Ell^p(\Omega)}$ and $u_\varepsilon \to u$ in $\Ell^p(\Omega)$ as $\varepsilon \downarrow 0$.
      Moreover, $u_\varepsilon \in \CC^\infty(\Omega)$.
      \item If $u \in \Ell_{\loc}^1(\Omega)$, then $u_\varepsilon \in \CC^\infty(\Omega_\varepsilon)$.
  \end{enumerate}
  \label{thm:mollification}
\end{thm}

\begin{proof}
  \begin{enumerate}[a)]
    \item For $i \in \{1, \dots, n\}$ and $h \in \R \setminus\{0\}$ let
    $$
    \DD^h_i v(x) \coloneqq \frac{1}{h} ( v(x + h e_i) - v(x)), \quad x \in \R^n.
    $$
    As $\eta_\varepsilon \in \CC_0^\infty(\R^n)$, we have $\nabla \eta_\varepsilon \in \Ell^\infty(\R^n)^n$.
    So $\DD_i^h  \eta_\varepsilon \in \Ell^\infty(\R^n)$ by the mean value theorem.
    As moreover $\DD_i^h \eta_\varepsilon(z) \to \frac{\partial \eta_\varepsilon}{\partial x_i}(z)$ as $h \to 0$ for any $z \in \R^n$, the dominated convergence theorem implies
    \begin{align*}
      \DD_i^h (u_\varepsilon)(x)
    &= \int_\Omega \frac{1}{h} (\eta_\varepsilon(x + h e_i - y) - \eta_\varepsilon (x - y)) u(y) \d y 
    = \int_\Omega (\DD_i^h \eta_\varepsilon (x - y)) u(y) \d y  \\
    &\to \int_\Omega \frac{\partial \eta_\varepsilon}{\partial x_i}(x - y) u(y) \d y \quad\text{as } h \downarrow 0.
    \end{align*}
    Hence, $\frac{\partial}{\partial x_i} u_\varepsilon(x) = \int_\Omega \frac{\partial \eta_\varepsilon}{\partial x_i} (x - y) u(y) \d y$.
    By induction, $u_\varepsilon \in \CC^\infty(\R^n)$.
    By \emph{Lebesgue's differentiation theorem} (see \cite[Section E.4]{evans2010partial}), we have
    \begin{equation}
      \lim_{r \downarrow 0} \frac{1}{|\BB_r(x)|} \int_{\BB_r(x)} |u(y) - u(x)| \d y = 0 \quad\text{for a.e. } x \in \Omega.
      \label{eq:lebdiffthm}
    \end{equation}
    For any such $x$ we obtain (by choosing $\varepsilon > 0$ small such that $\overline{\BB_\varepsilon(x)} \subset \Omega$)
    \begin{align}
      \begin{split}
      |u_\varepsilon(x) - u(x)|
      &= \Big| \int_{\BB_\varepsilon(x)} \eta_\varepsilon (x - y) u(y) \d y - u(x) \Big|  \\
      &= \Big| \int_{\BB_\varepsilon(x)} \eta_\varepsilon (x - y) (u(y) - u(x)) \d y \Big| \\
      &\leq \frac{1}{\varepsilon^n} \int_{\BB_\varepsilon(x)} \|\eta\|_{\Ell^\infty(\R^n)} |u(y) - u(x)| \d y \\
      &\leq \frac{C}{|\BB_\varepsilon(x)|} \int_{\BB_\varepsilon(x)} |u(y) - u(x) | \d y \label{eq:pointwise}
      \quad\to 0 \text{ as } \varepsilon \downarrow 0 
      \end{split}
    \end{align}
    due to \eqref{eq:lebdiffthm}.
    
    \item If $\supp(u) \Subset \Omega$, let $\delta \coloneqq \dist(\supp(u), \partial \Omega) > 0$.
    Then for any $x \in \Omega \setminus \Omega_{\frac{\delta}{2}}$ and $\varepsilon \leq \frac{\delta}{2}$ we have $\BB_\varepsilon(x) \cap \supp(u) = \emptyset$ and 
    $$
    u_\varepsilon(x) = \int_{\BB_\varepsilon(x) \cap \Omega} \eta_\varepsilon(x - y) u(y) \d y = 0.
    $$
    Hence, $\supp(u_\varepsilon) \subset \overline{\Omega_{\frac{\delta}{2}}} \Subset \Omega$.
    By a), $u_\varepsilon \in \CC_0^\infty(\Omega)$.

    \item For $u \in \CC^0(\Omega)$ and $V \Subset \Omega$, choose $W$ such that $V \Subset W \Subset \Omega$.
    Then $u$ is uniformly continuous in $W$ and \eqref{eq:lebdiffthm} holds uniformly for $x \in V$.
    Hence, also \eqref{eq:pointwise} is satisfied uniformly for $x \in V$ and thus $u_\varepsilon \to u$ uniformly in $V$.

    \item For $x \in \Omega$, by using Hölder's inequality and $\eta_\varepsilon \geq 0$ along with $\int_{\R^n} \eta_\varepsilon(z) \d y = 1$, we get
    \begin{align*}
      |u_\varepsilon(x)| 
      &= \Big| \int_\Omega \left( \eta_\varepsilon(x - y) \right)^{1 - \frac{1}{p}} \left( \eta_\varepsilon(x - y)\right)^{\frac{1}{p}} u(y) \d y \Big| \\
      &\leq \underbrace{\left( \int_\Omega \eta_\varepsilon(x -y) \d y\right)^{\frac{p - 1}{p}}}_{\leq 1} \left( \int_\Omega \eta_\varepsilon(x - y) |u(y)|^p \d y \right)^\frac{1}{p}
      \leq \left( \int_\Omega \eta_\varepsilon (x - y) |u(y)|^p \d y \right)^{\frac{1}{p}}.
    \end{align*}
    Raising this to the power of $p$ and integrating w.r.t $x \in \Omega$, by using Fubini we have
    \begin{align}
      \label{eq:lpinequality}
      \begin{split}
      \|u_\varepsilon\|_{\Ell^p(\Omega)}^p \|
      &\leq \int_\Omega\int_\Omega \eta_\varepsilon(x - y) |u(y)|^p \d x \d y
      = \int_\Omega |u(y)|^p \underbrace{\left( \int_\Omega \eta_\varepsilon(x -y) \d x \right)}_{\in [0,1]} \d y \\
      &\leq \int_\Omega |u(y)|^p \d y = \|u\|_{\Ell^p(\Omega)}^p.
      \end{split}
    \end{align}
    In particular, this implies $u_\varepsilon \in \Ell^p(\Omega)$.

    Given $\mu > 0$, we may choose $\varphi \in \CC_0^\infty(\Omega)$ such that $\|u - \varphi\|_{\Ell^p(\Omega)} <  \frac{\mu}{3}$. 
    As $\varphi$ and $\varphi_\varepsilon$ have compact support in $\Omega$ by b), we deduce from c) that $\varphi_\varepsilon \to \varphi$ uniformly in $\Omega$ as $\varepsilon \downarrow 0$.
    Hence, we may choose $\varepsilon_0 > 0$ small enough such that $\|\varphi_\varepsilon - \varphi\|_{\Ell^p(\Omega)} < \frac{\mu}{3}$ for all $\varepsilon \in (0, \varepsilon_0)$.
    But then,
    \begin{align*}
      \|u_\varepsilon - u\|_{\Ell^p(\Omega)}
      & \leq \|u_\varepsilon - \varphi_\varepsilon \|_{\Ell^p(\Omega)} + \|\varphi_\varepsilon - \varphi\|_{\Ell^p(\Omega)} + \|\varphi - u \|_{\Ell^p(\Omega)} \\
      &\leq \|\eta_\varepsilon \ast u - \eta_\varepsilon \ast \varphi \|_{\Ell^p(\Omega)} + \frac{2}{3} \mu \\
      &= \|\eta_\varepsilon \ast (u - \varphi) \|_{\Ell^p(\Omega)} + \frac{2}{3}\mu
      = \|(u - \varphi)_\varepsilon \|_{\Ell^p(\Omega)} + \frac{2}{3}\mu \\
      &\overset{\eqref{eq:lpinequality}}{\leq} \|u-\varphi\|_{\Ell^p(\Omega)} + \frac{2}{3}\mu < \mu \quad\text{for all } \varepsilon \in (0, \varepsilon_0).
    \end{align*}
    We still have $u_\varepsilon \in \CC^\infty(\Omega)$ since for $x \in \BB_\delta(x_0)$ with $\overline{\BB_{2\delta}(x_0)} \subset \Omega$ we have for 
    $x \in K \coloneqq \overline{\BB_{2\delta}(x_0)}$ and $\varepsilon \in (0, \delta)$, $u_\varepsilon(x) = \int_K \eta_\varepsilon(x - y)u(y) \d y$
    with $u \in L^1(K)$. Hence, the same argument as in a) shows $u_\varepsilon \in C^\infty (K)$ and therefore $u_\varepsilon \in C^\infty (\Omega)$ 
    as $\Omega$ is open and $x_0 \in \Omega$ was arbitrary.
    A similar argument shows e).\qedhere
  \end{enumerate}
\end{proof}

%%%%%%%%%%%%%%%%%%%%%%%%%%%%%%%%%%%%%%%%%%%%%%%%%%%%
%   Eigener Lösungsvorschlag
%    a): Let $h > 0$, $x \in \R^n$ and fix some $i = 1,\dots,n$. Since $u \in \Ell^1(\Omega)$, by extending $u$ with $0$ we can assume $u \in \Ell^1(\R^n)$. 
%   We denote this extension again with $u$.
%   Then, we have
%   $$
%   \frac{u_\varepsilon(x + h e_i) - u_\varepsilon(x)}{h}
%   = \int_{\R^n} \frac{\eta_\varepsilon(x + h e_i - y) - \eta_\varepsilon(x - y)}{h} u(y) \d y .
%   $$
%   The integrand on the right hand side of the above inequality is pointwise convergent to $\partial_i \eta_\varepsilon$.
%   Furthermore, by the mean value theorem and the fact that $\eta_\varepsilon \in \CC_0^\infty(\R^n)$ we have
%   $$
%   \left|\frac{\eta_\varepsilon(x + h e_i - y) - \eta_\varepsilon(x - y)}{h} u(y)\right|
% \leq \|\partial_i \eta_\varepsilon \|_\infty |u(y)|, \quad \forall y \in \R^n.
%   $$
%   But the right hand side is by assumption in $\Ell^1(\R^n)$, so Lebesgue's theorem yields
%   $$
%   \partial_i u_\varepsilon(x) = (\partial_i \eta_\varepsilon) \ast u (x),
%   $$
%   with $\partial_i \eta_\varepsilon \in \CC_0^\infty(\R^n)$.
%   This procedure can now be repeated yielding arbitrary derivatives of $u_\varepsilon$.

%   We now show that $u_\varepsilon(x) \to u(x)$ a.e. for $\varepsilon \downarrow 0$.
%   Since
%   \begin{align*}
%   |u_\varepsilon(x) - u(x)|
%   &\leq \left| \int_{\R^n} \eta_\varepsilon(x - y) u(y) \d y - u(x) \right| \\
%   &\overset{\mathrm{iii)}}{=} \left| \frac{1}{\varepsilon^n} \int_{\R^n} \eta\left(\frac{x - y}{\varepsilon}\right) ( u(y) - u(x) ) \d y \right| \\
%   &\leq \frac{C}{|\BB_\varepsilon(x)|} \int_{\BB_{\varepsilon}(x)} \|\eta\|_\infty |u(y) - u(x)| \d y,
%   \end{align*}
%   where we have used properties of the standard mollifier, the claim follows from the Lebesgue differentiation theorem.

%    b): If $\partial \Omega = \emptyset$, then $\Omega$ is clopen. 
%    But since $\R^n$ is connected, the only clopen sets are $\emptyset$ and $\R^n$.
%    In both cases, the claim follows by choosing $\varepsilon = 1$.
%    Hence, in the following we will assume that $\partial \Omega \neq \emptyset$.
%    Since $S: \supp(u)$ is compactly contained in $\Omega$ by assumption, the continuous function 
%    $$d(\cdot,\partial\Omega) \colon \supp(u) \to \R, x \mapsto \d(x,\partial\Omega) \inf_{z \in \partial \Omega} |z - x|$$
%    attains a minimum $\varepsilon_{\mathrm{min}}$. If we now choose $\varepsilon < \varepsilon_{\mathrm{min}}$, we deduce that
%    $$
%    \supp(u_\varepsilon) \subseteq \supp(u) + \overline{\BB_\varepsilon(0)} \subsetneq \Omega
%    $$
%    and the claim follows.

%   c): Note that $V \Subset \overline{V}_\gamma \Subset \Omega$ for some $\varepsilon > 0$ since $V$ is compactly contained in $\Omega$.
%   Since $u \in \CC^0(\Omega)$, this function is uniformly continuous on $\overline V_\gamma$. Hence, for all $\alpha > 0$ there exists $\beta > 0$ such that for all $x,y \in \overline V_\gamma$ with $|x - y| < \beta$
%   $$
%   |u(x) - u(y)| \leq \frac{\alpha}{C \|\eta\|_\infty},
%   $$
%   Then, with the same estimates as for a), for all $x \in V, y \in \overline{V}_\varepsilon$ and $\varepsilon \leq \gamma$ the following holds:
%   $$
%   |u_\varepsilon(x) - u(x)|
%   \leq \frac{C}{|\BB_\varepsilon(x)|} \int_{\BB_{\varepsilon}(x)} \|\eta\|_\infty |u(y) - u(x)| \d y \leq  \alpha.
%   $$
%   Therefore, the convergence on $V$ is uniform.

%   d): We extend $u$ with $0$ outside of $\Omega$ and calculate
%   \begin{align*}
%     \int_\Omega |u_\varepsilon(x)|^p \d x
%     &\leq \int_\Omega \left| \int_\Omega \eta_\varepsilon(x - y) u(y) \d y \right|^p \d x\\
%     &\leq \int_\Omega \left| \int_\Omega \eta_\varepsilon(x-y)^{1 - \frac{1}{p}}\eta_\varepsilon(x - y)^{\frac{1}{p}} u(y) \d y \right|^p \d x \\
%     &\leq \int_\Omega \left(\int_\Omega \eta_\varepsilon(x - y) \d y\right)^{1 - \frac{1}{p}} \left(\int_\Omega \eta_\varepsilon(x - y)|u(y)|^p \d y \right) \d x \\
%     &\leq \int_\Omega |u(y)|^p \left(\int_\Omega \eta_\varepsilon(x - y) \d x\right) \d y,
%   \end{align*}
%   where we used the Hölder inequality.
%   The right hand side is finite since by assumption $u \in \Ell^p(\Omega)$ thus $u_\varepsilon \in \Ell^p(\Omega)$ as well and we furthermore have
%   $$
%   \|u_\varepsilon\|_{\Ell^p(\Omega)} \leq \|u\|_{\Ell^p(\Omega)}
%   $$
%   considering that the integral of $\eta_\varepsilon$ is $1$ by iii).

%   Now let $\delta > 0$ be given.
%   We know that $C_0^0(\Omega)$ is dense in $\Ell^p(\Omega)$. 
%   Thus, there exists $\tilde u \in C_0^\infty(\Omega)$ with
%   $$
%   \| \tilde u - u \|_{\Ell^p(\Omega)} < \frac{\delta}{3}.
%   $$
%   Furthermore, since $\tilde u$ has compact support, by c) we know that that $\tilde u_\varepsilon$ converges uniformly to $\tilde u$.
% Therefore, there exists $\varepsilon' > 0$ such that for all $\varepsilon \leq \varepsilon'$ we have
%   $$
%   \|\tilde u_\varepsilon - \tilde u\|_{\Ell^p(\Omega)} \leq \|\tilde u_\varepsilon - \tilde u\|_{\Ell^\infty(\Omega)} \leq \frac{\delta}{3},
%   $$
%   where we used the fact that $\tilde u$ and $\tilde u_\varepsilon$ have compact support to estimate the $\Ell^p$-norms.
%   Finally, we have by the previous calculations that
%   $$
%   \|u_\varepsilon - \tilde u_\varepsilon\|_{\Ell^p(\Omega)}
%   = \|(u - \tilde u)_\varepsilon\|_{\Ell^p(\Omega)}
%   \leq \|u - \tilde u\|_{\Ell^p(\Omega)} \leq \frac{\delta}{3}.
%   $$
%   We can now deduce
%   $$
%   \|u_\varepsilon - u\|_{\Ell^p(\Omega)} \leq \delta
%   $$
%   by the usual argument and conclude that $u_\varepsilon \to u$ in $\Ell^p$.

%   e): Extending $u$ by $0$ outside of $\Omega$ we can assume $u \in \Ell^1_{\loc}(\R^n)$.
%   Let $\varepsilon > 0$, $x \in \Omega_\varepsilon$ and fix $i =1,\dots,n$. 
%   Choose $h> 0$ such that $x + he_i \in \Omega_\varepsilon$. Like in a) we calculate
%   \begin{align*}
%   \frac{u_\varepsilon(x + h e_i) - u_\varepsilon(x)}{h}
%   &= \int_{\R^n} \frac{\eta_\varepsilon(x + h e_i - y) - \eta_\varepsilon(x - y)}{h} u(y) \d y, \\
%   \intertext{where the right hand side is well defined, since the difference of two $\CC_0^\infty(\R^n)$ functions has again compact support and we furthermore deduce}
%   &= \int_{\BB_{\varepsilon + h}(x)} \frac{\eta_\varepsilon(x + h e_i - y) - \eta_\varepsilon(x - y)}{h} u(y) \d y.
% \end{align*}
%   We can now argue as in a) to prove the differentiability.
%%%%%%%%%%%%%%%%%%%%%%%%%%%%%%%%%%%%%%%%%%%%%%%%%%%%%%%%


\section{Polar Coordinates}
\label{sec:polar}

Let $f \in \CC^1(\overline{\BB_r(x_0)})$ with $x_0 \in \R^n$, $r > 0$. Then by the transformation rule with $x = x_0 + sz$, $s \in (0,r)$, $z \in \partial \BB_1(0)$ we have
$$
\int_{\BB_r(x_0)} f(x) \d x = \int_0^r \left( \int_{\partial \BB_s(x_0)} f \d \sigma(x) \right) \d s = \int_0^r s^{n - 1} \int_{\partial \BB_1(0)} f(x_0 + sz) \d \sigma(z) \d s.
$$
In particular, if $x_0 = 0$, $f$ is radially symmetric and $\omega_n$ is the surface $|\partial \BB_1(0)|$ of $\partial \BB_1(0)$, we get
$$
\int_{\BB_r(0)} f(x) \d x = \omega_n \int_0^r f(s) s^{n - 1} \d s,
$$
where $s = |x|$.

\begin{proof}
  See \cite[Section C.3]{evans2010partial}.
\end{proof}
