\section{Weak derivatives and definitions of Sobolev spaces}

We already saw in the introduction that for $u \in \CC^1(\Omega)$ with $\Omega \subset \R^n$ open, we have 
\begin{equation}
  \int_\Omega u(x) \varphi_{x_i}(x) \d x = -\int_\Omega u_{x_i}(x) \varphi(x) \d x, \quad \forall \varphi \in \CC_0^\infty(\Omega).
\end{equation}

More generally, for higher order derivatives we have the following result:
\begin{lem}
  Let $\Omega \subset \R^n$ be open, $u \in \CC^k(\Omega)$ with $k \in \N$, and $\alpha \in \N_0^n$ be a multiindex with $|\alpha| = k$. Then
  \begin{equation}
    \int_\Omega u(x) \DD^\alpha \varphi(x) = (-1)^{|\alpha|} \int_\Omega \DD^\alpha u(x) \varphi(s) \d x, \quad\forall \varphi \in \CC_0^\infty(\Omega).
  \end{equation}
\end{lem}

\begin{proof}
  For $k = 1$, (3.2) is just (3.1) which was verified in the exercise.
  For $\alpha = (\alpha_1,\dots,\alpha_n)$ with $|\alpha| = k$ we have
  $$
  \DD^\alpha \phi(x)= \frac{\partial^{\alpha_1}}{\partial x_1^{\alpha_1}} ( \dots ( \frac{\partial^{\alpha_n}}{\partial x_n^{\alpha_n}} ) \dots )(x)
  $$
  and (3.2) follows by applying (3.1) $k$ times.
\end{proof}

In order to define the weak derivative $\DD^\alpha u$, we look for a variant of (3.2) which is satisfied if $u$ has less regularity than being in $\CC^k(\Omega)$.
As the integrals in (3.2) are meaningful if $u, \DD^\alpha u \in \Ell^1_{\loc}(\Omega)$, we define the weak derivative $\DD^\alpha u$ of $u$ as follows (see introduction for $|\alpha| = 1$).

\begin{defn}
  Let $\Omega$ be an open set, $u \in \Ell^1_{\loc}(\Omega)$ and $\alpha \in \N_0^n$ a multiindex. 
  $u$ has the \emph{$\alpha$th weak partial derivative} $\DD^\alpha u$ if there is $v \in \Ell_{\loc}^1(\Omega)$ such that
  \begin{equation}
    \int_\Omega u(x) \DD^\alpha \varphi(x) \d x = (-1)^{|\alpha|} \int_\Omega v(x) \varphi(x) \d x, \quad \forall \varphi \in \CC_c^\infty(\Omega).
  \end{equation}
\end{defn}

If (3.3) is satisfied, we define $\DD^\alpha u \coloneqq v$.
In order to show the uniqueness of the weak derivative we need the following fundamental lemma.

\begin{lem}[Fundamental lemma of calculus of variations]
  Let $\Omega \subset \R^n$ be open and $u \in \Ell_{\loc}^1(\Omega)$.
  Then we have the equivalence
  $$
  \int_\Omega u(x) \varphi(x) \d x = 0, \quad \forall \varphi \in \CC_0^\infty(\Omega) \iff u = 0 \text{ a.e. in } \Omega.
  $$
\end{lem}

\begin{proof}
  ``$\impliedby$'' is obvious.

``$\implies$'': Let $u \in \Ell^1_{\loc}(\Omega)$ with $\int_\Omega u\varphi \d x = 0, \forall \varphi \in \CC_0^\infty(\Omega)$.
We fix $K \subset \Omega$ compact and define
$$
\operatorname{sign}(u(x)) \coloneqq 
\begin{cases} 
  1, &\text{ if } u(x) > 0, \\
  -1, &\text{ if } u(x) <  0, \\
  0, &\text{ if } u(x) \in \{ 0, -\infty, +\infty\} \\
\end{cases}
$$
and
$$
f(x) \coloneqq \begin{cases} \operatorname{sign}(u(x)), &\text{ if } x \in K, \\ 0,  &\text{ if } x \in \R^n \setminus K. \end{cases}
$$
As $|u| < \infty$ a.e. in $K$ with $\supp(f) \subset K \Subset \Omega$, we define $\varphi_n \coloneqq f_{\frac{1}{n}} = \eta_{\frac{1}{n}} \ast f$ and deduce from Theorem 2.1 a), b) that $\varphi_n \in \CC_0^\infty(\Omega)$ and $\varphi_{n_k}(x) \to f(x)$ a.e. in $\Omega$ as $k \to \infty$ for some subsequence.
As moreover
$$
|\varphi_{n_k}(x)| 
\leq \int_\Omega \eta_{\frac{1}{n}}(x - y) |f(y)| \d y
\leq \underbrace{\vphantom{\int_\Omega}\|f\|_{\Ell^\infty(\Omega)}}_{\leq 1} \underbrace{ \int_\Omega \eta_{\frac{1}{n}} (x - y) \d y}_{\leq 1}
\leq 1, \quad \forall x \in \Omega, k \in \N,
$$
the dominated convergence theorem implies
$$
0 = \lim_{k \to \infty} \int_\Omega u(x) \varphi_{n_k}(x) \d x 
= \int_\Omega u(x) f(x) \d x = \int_K |u(x)| \d x.
$$
Hence, $u = 0$ a.e. in $K$.
As e.g. $\Omega = \bigcup_{k = 1}^\infty K_n$ with $K_n \coloneqq \overline{\Omega_{\frac{1}{n}}} \cap \overline{\BB_n(0)}$ and $u = 0$ a.e. in $K_n$ (as $K_n \subset \Omega$ compact), we have $u = 0$ a.e. in $\Omega$.
\end{proof}

With this result we show the uniqueness of the weak derivative and its equality with the classical derivative if $u$ is classically differentiable.

\begin{lem}
  Let $u \in \Ell_{\loc}^1(\Omega)$ and $\alpha \in \N_0^n$, with $|\alpha| = k \in \N$. 
  If the weak derivative $\DD^\alpha u(\Omega)$ exists it is uniquely defined up to a set of measure zero.
  If $u \in \CC^k(\Omega)$, then $\DD^\alpha u$ exists and is equal to the classical derivative $\DD^\alpha u$. 
  Hence, we use $\DD^\alpha$ both for weak and classical partial derivatives.
\end{lem}

\begin{proof}
  Simple.
\end{proof}
