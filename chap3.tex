\chapter{Weak derivatives and definitions of Sobolev spaces}

We already saw in the introduction that for $u \in \CC^1(\Omega)$ with $\Omega \subset \R^n$ open, we have 
\begin{equation}
  \label{eq:partialIntegration}
  \int_\Omega u(x) \varphi_{x_i}(x) \d x = -\int_\Omega u_{x_i}(x) \varphi(x) \d x, \quad \text{ for all } \varphi \in \CC_0^\infty(\Omega).
\end{equation}

More generally, for higher order derivatives we have the following result:
\begin{lem}
  \label{lem:partialIntegrationGeneral}
  Let $\Omega \subset \R^n$ be open, $u \in \CC^k(\Omega)$ with $k \in \N$, and $\alpha \in \N_0^n$ be a multiindex with $|\alpha| = k$. Then
  \begin{equation}
    \label{eq:partialIntegrationGeneral}
    \int_\Omega u(x) \DD^\alpha \varphi(x) = (-1)^{|\alpha|} \int_\Omega \DD^\alpha u(x) \varphi(s) \d x, \quad\text{for all } \varphi \in \CC_0^\infty(\Omega).
  \end{equation}
\end{lem}

\begin{proof}
  For $k = 1$, (\ref{eq:partialIntegrationGeneral}) is just (\ref{eq:partialIntegration}) which was verified in the exercise.
  For $\alpha = (\alpha_1,\dots,\alpha_n)$ with $|\alpha| = k$ we have
  $$
  \DD^\alpha \phi(x)= \frac{\partial^{\alpha_1}}{\partial x_1^{\alpha_1}} ( \dots ( \frac{\partial^{\alpha_n}}{\partial x_n^{\alpha_n}} ) \dots )(x)
  $$
  and (\ref{eq:partialIntegrationGeneral}) follows by applying (\ref{eq:partialIntegration}) $k$ times.
\end{proof}

In order to define the weak derivative $\DD^\alpha u$, we look for a variant of (\ref{eq:partialIntegrationGeneral}) which is satisfied if $u$ has less regularity than being in $\CC^k(\Omega)$.
As the integrals in (\ref{eq:partialIntegrationGeneral}) are meaningful if $u, \DD^\alpha u \in \Ell^1_{\loc}(\Omega)$, we define the weak derivative $\DD^\alpha u$ of $u$ as follows (see introduction for $|\alpha| = 1$).

\begin{defn}
  \label{defn:weakPartialDerivative}
  Let $\Omega$ be an open set, $u \in \Ell^1_{\loc}(\Omega)$ and $\alpha \in \N_0^n$ a multiindex. 
  $u$ has the \emph{$\alpha$th weak partial derivative} \index{weak partial derivative} $\DD^\alpha u$ if there is $v \in \Ell_{\loc}^1(\Omega)$ such that
  \begin{equation}
    \label{eq:weakPartialDerivativeDef}
    \int_\Omega u(x) \DD^\alpha \varphi(x) \d x = (-1)^{|\alpha|} \int_\Omega v(x) \varphi(x) \d x, \quad \text{for all } \varphi \in \CC_0^\infty(\Omega).
  \end{equation}
\end{defn}

If (\ref{eq:weakPartialDerivativeDef}) is satisfied, we define $\DD^\alpha u \coloneqq v$.
In order to show the uniqueness of the weak derivative we need the following fundamental lemma.

\begin{lem}[Fundamental lemma of calculus of variations]
  \label{lem:fundamental}
  Let $\Omega \subset \R^n$ be open and $u \in \Ell_{\loc}^1(\Omega)$.
  Then we have the equivalence
  $$
  \int_\Omega u(x) \varphi(x) \d x = 0, \quad\text{for all } \varphi \in \CC_0^\infty(\Omega) \iff u = 0 \text{ a.e. in } \Omega.
  $$
\end{lem}

\begin{proof}
  ``$\impliedby$'' is obvious.

  ``$\implies$'': Let $u \in \Ell^1_{\loc}(\Omega)$ with $\int_\Omega u\varphi \d x = 0, \text{ for all } \varphi \in \CC_0^\infty(\Omega)$.
We fix $K \subset \Omega$ compact and define
$$
\operatorname{sign}(u(x)) \coloneqq 
\begin{cases} 
  1, &\text{ if } u(x) > 0, \\
  -1, &\text{ if } u(x) <  0, \\
  0, &\text{ if } u(x) \in \{ 0, -\infty, +\infty\} \\
\end{cases}
$$
and
$$
f(x) \coloneqq \begin{cases} \operatorname{sign}(u(x)), &\text{ if } x \in K, \\ 0,  &\text{ if } x \in \R^n \setminus K. \end{cases}
$$
As $|u| < \infty$ a.e. in $K$ with $\supp(f) \subset K \Subset \Omega$, we define $\varphi_n \coloneqq f_{\frac{1}{n}} = \eta_{\frac{1}{n}} \ast f$ and deduce from Theorem \ref{thm:mollifier} a), b) that $\varphi_n \in \CC_0^\infty(\Omega)$ and $\varphi_{n_k}(x) \to f(x)$ a.e. in $\Omega$ as $k \to \infty$ for some subsequence.
As moreover
$$
|\varphi_{n_k}(x)| 
\leq \int_\Omega \eta_{\frac{1}{n}}(x - y) |f(y)| \d y
\leq \underbrace{\vphantom{\int_\Omega}\|f\|_{\Ell^\infty(\Omega)}}_{\leq 1} \underbrace{ \int_\Omega \eta_{\frac{1}{n}} (x - y) \d y}_{\leq 1}
\leq 1, \quad\text{for all } x \in \Omega, k \in \N,
$$
the dominated convergence theorem implies
$$
0 = \lim_{k \to \infty} \int_\Omega u(x) \varphi_{n_k}(x) \d x 
= \int_\Omega u(x) f(x) \d x = \int_K |u(x)| \d x.
$$
Hence, $u = 0$ a.e. in $K$.
As e.g. $\Omega = \bigcup_{k = 1}^\infty K_n$ with $K_n \coloneqq \overline{\Omega_{\frac{1}{n}}} \cap \overline{\BB_n(0)}$ and $u = 0$ a.e. in $K_n$ (as $K_n \subset \Omega$ compact), we have $u = 0$ a.e. in $\Omega$.
\end{proof}

With this result we show the uniqueness of the weak derivative and its equality with the classical derivative if $u$ is classically differentiable.

\begin{lem}
  Let $u \in \Ell_{\loc}^1(\Omega)$ and $\alpha \in \N_0^n$, with $|\alpha| = k \in \N$. 
  If the weak derivative $\DD^\alpha u(\Omega)$ exists it is uniquely defined up to a set of measure zero.
  If $u \in \CC^k(\Omega)$, then $\DD^\alpha u$ exists and is equal to the classical derivative $\DD^\alpha u$. 
  Hence, we use $\DD^\alpha$ both for weak and classical partial derivatives.
\end{lem}

\begin{proof}
  If $v$ and $\tilde v$ are $\alpha$th weak derivatives of $u$, by (\ref{eq:partialIntegrationGeneral})
  $$
  \int_\Omega (v - \tilde v) (x) \varphi(x) \d x = 0, \quad\text{for all } \varphi \in \CC_0^\infty(\Omega).
  $$
  Hence, by Lemma \ref{lem:fundamental} $v - \tilde v = 0$ a.e. in $\Omega$ and $v = \tilde v$ a.e. in $\Omega$.
  If $u \in \CC^k(\Omega)$, then by Lemma \ref{lem:partialIntegrationGeneral}, (\ref{eq:partialIntegrationGeneral}) is satisfied with $v = \DD^\alpha u$ and hence the classical derivative $\DD^\alpha u$ is also a weak derivative.
  Due to the uniqueness the claim follows.
\end{proof}

As the weak derivative is well-defined, we may now define Sobolev spaces consisting of functions having weak derivatives in $\Ell^p$ spaces.

\begin{defn}
  \begin{enumerate}[a)]
    \item Let $k \in \N$, $p \in [1,\infty]$ and $\Omega \subset \R^n$ be open. 
      We define the Sobolev space
      $$
      \WW^{k,p}(\Omega) \coloneqq \bigg\{u \in \Ell^p(\Omega) \colon \text{ weak derivative } D^\alpha u \text{ ex. with } \DD^\alpha u \in \Ell^p(\Omega), \text{ for all } 0 \leq |\alpha| \leq k \bigg\}
      $$
      with the norm
      $$
      \|u\|_{\WW^{k,p}(\Omega)} \coloneqq 
      \begin{cases}
        \left( \sum_{|\alpha| \leq k} \|\DD^\alpha u\|_{\Ell^p(\Omega)}^p \right)^{\frac{1}{p}} &\text{ if } p \in [1,\infty), \\
          \sum_{|\alpha| \leq k} \|\DD^\alpha u\|_{\Ell^\infty(\Omega)} &\text{ if } p = \infty.
      \end{cases}
      $$
      We further define
      $$
      \WW_0^{k,p}(\Omega) \coloneqq \overline{\CC_0^\infty(\Omega)}^{\|\cdot\|_{\WW^{k,p}(\Omega)}}
      $$
      to be the closure of $\CC_0^\infty(\Omega)$ in $\WW^{k,p}(\Omega)$ and
      $$
      \WW_{\loc}^{k,p}(\Omega) \coloneqq \bigcap_{V \Subset \Omega} W^{k,p}(V).
      $$
      For $p = 2$, we define $\HH^k(\Omega) \coloneqq \WW^{k,2}(\Omega)$ and $\HH_0^k(\Omega) \coloneqq \WW_0^{k,2}(\Omega)$.
    \item For $(u_m)_{m \in \N} \subset \WW^{k,p}(\Omega)$ and $u \in \WW^{k,p}(\Omega)$, we say $u_m \to u$ in $W^{k,p}(\Omega)$ if $$\lim_{n \to \infty} \|u_m - u\|_{\WW^{k,p}(\Omega)} = 0.$$  
      We say $u_m \to u$ in $\WW_{\loc}^{k,p}(\Omega)$ if $u_m \to u$ in $\WW^{k,p}(V)$ for all $V \Subset \Omega$.
  \end{enumerate}
\end{defn}

\begin{rem}
  \begin{enumerate}[a)]
    \item For $\alpha = (0,\dots,0)$ we set $\DD^\alpha u = \DD^0 u = u$.
      We further identify functions in $\WW^{k,p}(\Omega)$ which agree a.e.
      If for $u \in \WW^{k,p}(\Omega)$ the equivalence class $[u]$ contains a continuous representative, the latter is chosen for $u$.
    \item $u \in \WW_0^{k,p}(\Omega)$ if and only if there exists $(u_m)_{m \in \N} \subset \CC_0^\infty(\Omega)$ such that $u_m \to u$ in $\WW^{k,p}(\Omega)$.
      We interpret $W_0^{k,p}(\Omega)$ as the set of $u \in W^{k,p}(\Omega)$ such that ``$\DD^\alpha u = 0$ on $\partial \Omega$ for any $|\alpha| \leq k - 1$''.
      This interpretation will be made precise in Chapter \ref{chap:extAndTrace}.
    \item The letter $\HH$ in $\HH^k(\Omega)$ and $\HH_0^k(\Omega)$ is used as those are Hilbert spaces as we will see soon.
  \end{enumerate}
\end{rem}

\begin{ex}
  Let $\Omega = \BB_1(0) \subset \R^n$, $u(x) = |x|^{-a}$ for $x \in \Omega \setminus \{0\}$ with some $a > 0$.
  Given $p \in [1,\infty)$, for which $a$ do we have $u \in \WW^{1,p}(\Omega)$?

    Since $u \in C^\infty(\Omega \setminus \{0\})$, we have for $x \neq 0$
    \begin{align*}
      u_{x_i}(x) &= -a|x|^{-a-1} \frac{x_i}{|x|} = -\frac{ax_i}{|x|^{a + 2}} \quad \text{and} \\
      |\nabla u(x)| &= \frac{a}{|x|^{a + 1}}.
    \end{align*}
    For fixed $\varphi \in \CC_0^\infty(\Omega)$ and $\varepsilon > 0$, Green's formula ($\nu$ is the outward unit normal on $\Omega \setminus \overline{\BB_\varepsilon(0)}$ ) implies
    \begin{equation}
      \label{eq:greenFormulaEx}
      \int_{\Omega \setminus \overline{\BB_\varepsilon(0)}} u\varphi_{x_i} \d x
      = -\int_{\Omega \setminus \overline{\BB_\varepsilon(0)}} u_{x_i} \varphi \d x + \int_{\partial \Omega} u\varphi \nu_i \d \sigma
      + \int_{\partial \BB_\varepsilon(0)} u\varphi \nu_i \d \sigma.
    \end{equation}
    We may pass to the limit $\varepsilon \downarrow 0$ in the first two integrals if $u \in \Ell^1(\Omega)$ and $\nabla u \in \Ell^1(\Omega)^n$, i.e. $a < n$ and $a + 1 < n$.
    As for $a < n - 1$ we further have
    $$
    \left| \int_{\partial \BB_\varepsilon(0)} u\varphi \nu_i \d \sigma \right|
      \leq \|\varphi\|_{\Ell^\infty} \int_{\partial \BB_\varepsilon(0)} \varepsilon^{-a} \d \sigma
      \leq C \varepsilon^{k - 1 - a} \to 0 \text{ as } \varepsilon \downarrow 0.
    $$
    Hence, for $a < n - 1$ we may pass to the limit $\varepsilon \downarrow 0$ in (\ref{eq:greenFormulaEx}) and obtain $\int_\Omega u\varphi_{x_i} \d x = - \int_\Omega u_{x_i} \varphi \d x$.
    Hence, the weak derivative $u_{x_i}$ exists for $a < n - 1$.
    Hence, $u \in \WW^{1,p}(\Omega)$ if $u \in \Ell^p(\Omega)$ and $\nabla u = \frac{-a x}{|x|^{a + 2}} \in \Ell^p(\Omega)^n$, i.e. $ap < p$ and $(a + 1)p < n$.
    We conclude that
    $$
    u \in \WW^{1,p}(\Omega) \iff a < \frac{n - p}{p} \text{ (and $p < n$)}.
    $$
\end{ex}

Next, we prove some elementary properties of weak derivatives which are well known in the case of classical derivatives.

\begin{prop}
  \label{prop:sobolevProperties}
  Let $\Omega$ be open, $k \in \N$,$p \in [1,\infty]$, $u,v \in \WW^{k,p}(\Omega)$, and $\alpha \in \N_0^n$ with $1 \leq |\alpha| \leq k$.
  \begin{enumerate}[a)]
    \item $\DD^\alpha u \in W^{k - |\alpha|,p}(\Omega)$ (with $\WW^{0,p}(\Omega) = \Ell^p(\Omega)$) and $\DD^\beta (\DD^\alpha (u)) = \DD^\alpha (\DD^\beta(u)) = \DD^{\alpha + \beta}(u)$ for all $\alpha, \beta \in \N_0^n$ with $|\alpha| + |\beta| \leq k$.
    \item For $\lambda, \mu \in \R$ we have $\lambda u + \mu v \in \WW^{k,p}(\Omega)$ and $\DD^\alpha(\lambda u + \mu v) = \lambda \DD^\alpha u + \mu \DD^\alpha v$.
    \item If $V \subset \Omega$ is open, the $u \in \WW^{k,p}(V)$.
    \item If $\xi \in \CC_0^\infty(\Omega)$, then $\xi u \in \WW^{k,p}(\Omega)$ and Leibniz's formula
      $$
      \DD^\alpha(\xi u) = \sum_{\beta \leq \alpha} \binom{\alpha}{\beta} \DD^\beta\xi \DD^{\alpha - \beta} u
      $$
      holds with 
      $$\binom{\alpha}{\beta} = \frac{\alpha!}{(\alpha -\beta)! \beta!}, \quad \alpha! = \prod_{i = 1}^n \alpha_i!$$
      and
      $$\quad \beta \leq \alpha \iff \forall i \in \{1,\dots,n\}\colon \beta_i \leq \alpha_i.$$
  \end{enumerate}
\end{prop}

\begin{proof}
  b) and c) easily follow from Definition \ref{defn:weakPartialDerivative}.

  \begin{enumerate}
  \item[a)] Let $\varphi \in \CC_0^\infty(\Omega)$. 
    Then $\DD^\beta \varphi \in \CC_0^\infty(\Omega)$ and (\ref{eq:weakPartialDerivativeDef}) implies
  \begin{align*}
    \int_\Omega \DD^\alpha u \DD^\beta \varphi \d x
    &= (-1)^{|\alpha|} \int_\Omega u \DD^{\alpha + \beta} \varphi \d x \\
    &= (-1)^{|\alpha|} (-1)^{|\alpha| + |\beta|} \int_\Omega \DD^{\alpha + \beta} u \varphi \d x \\
    &= (-1)^{|\beta|} \int_\Omega \DD^{\alpha + \beta} u \varphi \d x,
  \end{align*}
  as $|\alpha| + |\beta| = |\alpha + \beta|$.
      Hence, $\DD^\beta(\DD^\alpha u) = \DD^{\alpha + \beta} u$, for $|\beta| \leq k - |\alpha|$.

    \item[d)] Let $\varphi \in \CC_0^\infty(\Omega)$. 
      In case of $|\alpha| = 1$, we have
      $$
      \int_\Omega \xi u \DD^\alpha \varphi \d x = \int_\Omega \left( u \DD^\alpha(\xi \varphi) - u(\DD^\alpha \xi) \varphi \right) \d x \overset{(3.3)}{=} -\int_\Omega \left( \xi \DD^\alpha u + u \DD^\alpha \xi \right) \varphi \d x.
      $$
      Hence, $\DD^\alpha(\xi u) = \xi \DD^\alpha u + u \DD^\alpha \xi \in \Ell^p(\Omega)$ and the claim is true for $|\alpha| = 1$.

      Assume the claim is true for all $|\alpha| \leq l$ with some $l \in \{1,\dots,k-1\}$ (IA).

      Let $\alpha$ satisfy $|\alpha| = l+1$.
      Then $\alpha  = \beta + \gamma$ for some $\beta, \gamma \in \N_0^k$ with $|\beta| = l$ and $|\gamma| = 1$.
      Hence,
      \begin{align*}
        \int_\Omega \xi u \DD^\alpha \varphi
        &= \int_\Omega \xi u\DD^\beta (\DD^\gamma \varphi ) \d x \\
        &\overset{\mathrm{(IA)}}{=} (-1)^{|\beta|} \int_\Omega \sum_{\sigma \leq \beta} \binom{\beta}{\sigma} \DD^\sigma \xi\, \DD^{\beta - \sigma} u\, \DD^\gamma \varphi \d x \\
        &\overset{(3.3)}{=} (-1)^{|\beta|+|\gamma|} \int_\Omega \sum_{\sigma \leq \beta} \binom{\beta}{\sigma} \DD^\gamma \left( \DD^\sigma \xi \, \DD^{\beta - \sigma} u \right) \varphi \d x \\
        &\overset{\mathrm{(IA)}}{=} (-1)^{|\alpha|} \int_\Omega \sum_{\sigma \leq \beta} \binom{\beta}{\sigma} \left[ \DD^{\sigma + \gamma} \xi\, \DD^{\beta - \sigma} u + \DD^\sigma \xi \,\DD^{\alpha - \sigma} u \right] \varphi \d x\\
        &= (-1)^{|\alpha|} \int_\Omega \sum_{\sigma \leq \beta} \binom{\beta}{\sigma} \left[ \DD^{\sigma + \gamma} \xi\, \DD^{\alpha - (\sigma + \gamma)} u + \DD^\sigma \xi \,\DD^{\alpha - \sigma} u \right] \varphi \d x\\
        &= (-1)^{|\alpha|} \int_\Omega \left [\sum_{\gamma \leq \rho \leq \alpha} \binom{\beta}{\rho - \gamma} + \sum_{0 \leq \rho \leq \beta} \binom{\beta}{\rho} \right] \DD^\rho \xi \,\DD^{\alpha - \rho}u \, \varphi \d x \\
        &= (-1)^{|\alpha|} \int_\Omega \sum_{\rho \leq \alpha} \left[ \binom{\beta}{\rho - \gamma} + \binom{\beta }{\rho}\right] \DD^\rho \xi \, \DD^{\alpha - \rho} u \, \varphi \d x
      \end{align*}
      with the convention $\binom{\beta}{\tilde\beta} = 0$ if $\beta_i < \tilde \beta_i$ or $\tilde \beta_i < 0$ for some $i \in \{1,\dots,n\}$.
      As $\binom{\beta}{\rho - \gamma} + \binom{\beta}{\rho} = \binom{\beta + \gamma}{\rho} = \binom{\alpha}{\rho}$, we deduce that the claim holds by induction.\qedhere
  \end{enumerate}
\end{proof}

Finally, we show that $\WW^{k,p}$ is a Banach space.

\begin{thm}
  Let $\Omega \subset \R^n$ be open, $k \in \N$, $p \in [1,\infty]$.
  Then $\WW^{k,p}(\Omega)$ is a Banach space.
  Moreover, $\HH^k(\Omega)$ is a Hilbert space with the scalar product
  $$
  (u,v)_{\HH^k(\Omega)} \coloneqq \sum_{|\alpha| \leq k} \int_\Omega \DD^\alpha u \overline{\DD^\beta v} \d x
  $$
\end{thm}

\begin{proof}
  By Proposition \ref{prop:sobolevProperties} b), $\WW^{k,p}$ is a vector space.
  For $p \in [1,\infty)$ and $u,v \in \WW^{k,p}(\Omega)$, Minkowski's inequality (see \ref{sec:lpBasics}) on $\Ell^p(\Omega)$ and for $\|\cdot\|_p$ on $\R^m$ implies
    \begin{align*}
      \|u + v\|_{W^{k,p}(\Omega)}
      &= \left( \sum_{|\alpha| \leq k} \|\DD^\alpha u + \DD^\alpha v \right)^{\frac{1}{p}} 
      \leq \left( \sum_{|\alpha|\leq k} \left( \|\DD^\alpha u\|_{\Ell^p(\Omega)} + \|\DD^\alpha v\|_{\Ell^p(\Omega)} \right)^p \right) \\
      &\leq \left( \sum_{|\alpha| \leq k} \|\DD^\alpha u\|_{\Ell^p(\Omega)}^p\right)^{\frac{1}{p}}
      + \left( \sum_{|\alpha| \leq k} \|\DD^\alpha v\|_{\Ell^p(\Omega)}^p\right)^{\frac{1}{p}}
      = \|u\|_{\WW^{k,p}(\Omega)} + \|v\|_{\WW^{k,p}(\Omega)}.
    \end{align*}
    All other properties of the norm are easily verified for $\|\cdot\|_{\WW^{k,p}(\Omega)}$.

    Let $(u_m)_{m \in \N}$ be a Cauchy sequence in $\WW^{k,p}(\Omega)$.
    Then for all $|\alpha|\leq k$, $(\DD^\alpha u_m)_{m \in \N}$ is a Cauchy sequence in $\Ell^p(\Omega)$ as 
    $
    \|\DD^\alpha u_m - \DD^\alpha u_l \|_{\Ell^p(\Omega)} \leq \|u_m - u_l \|_{\WW^{k,p}(\Omega)}.
    $
    Hence, there exists $u_\alpha \in \Ell^p(\Omega)$ with
    \begin{equation}
      \label{eq:derivativeConv}
      \DD^\alpha u_m \to u_\alpha \text{ in } \Ell^p(\Omega), |\alpha| \leq k.
    \end{equation}
    For $\alpha = (0,\dots,0)$ we define $u_{(0,\dots,0)}  \eqqcolon u$ and have
    \begin{equation}
      \label{eq:funcConv}
      u_m \to u \text{ in } \Ell^p(\Omega).
    \end{equation}
    To show that $u_\alpha = \DD^\alpha u$, we fix $\varphi \in \CC_0^\infty(\Omega)$ and obtain
    \begin{align*}
      \int_\Omega u \DD^\alpha \varphi \d x
      \overset{(\ref{eq:funcConv})}{=} \lim_{m \to \infty} \int_\Omega u_m \DD^\alpha \varphi \d x 
      = \lim_{m \to \infty} (-1)^{|\alpha|} \int_\Omega \DD^\alpha u_m \varphi \d x 
      \overset{(\ref{eq:derivativeConv})}{=} (-1)^{|\alpha|} \int_\Omega u_\alpha \varphi \d x.
    \end{align*}
    Since $\varphi, \DD^\alpha \varphi \in \Ell^q(\Omega)$ for all $|\alpha| \leq k$ and $u \in \WW^{k,p}(\Omega)$.
    But then (\ref{eq:derivativeConv}) and (\ref{eq:funcConv}) imply $u_m \to u$ in $\WW^{k,p}(\Omega)$.
    Hence, $\WW^{k,p}(\Omega)$ is complete and a Banach space.

    That $(\cdot,\cdot)_{\HH^k(\Omega)}$ is a scalar product on $\HH^k(\Omega)$ easily follows from the $\Ell^2$-scalar product.
    Hence, $\HH^k(\Omega)$ is a Hilbert space.
\end{proof}

In particular, $\WW_0^{k,p}(\Omega)$ is a Banach space and a subspace of $\WW^{k,p}(\Omega)$.
