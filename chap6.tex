\chapter{Embeddings and Sobolev Inequalities}

We ask the question if $u \in \WW^{k,p}(\Omega)$ automatically belongs to some other function spaces.
The answer is \emph{yes}, but we will see that to which spaces $u$ belongs depends on $p$, e.g. for $k = 1$ and $p < n$, $p = n$ or $p > n$.

More precisely we ask whether $X = \WW^{k,p}(\Omega)$ is embedded into a space $Y$ in the following sense.

\begin{defn}
  Let $X, Y$ be Banach spaces with $X \subset Y$.
  \begin{enumerate}[a)]
    \item We say that $X$ is \emph{continuously embedded into $Y$} if there is $C > 0$ such that
      $$
      \|u\|_Y \leq C \, \|u\|_X \quad\text{for all } u \in X.
      $$
    \item We say that $X$ is \emph{compactly embedded into $Y$} if it is continuously embedded into $Y$ and any bounded sequence in $X$ is \emph{precompact in $Y$}, i.e. for any $(u_m)_{m \in \N} \subset X$ with $\|u_m\|_X \leq M$ for all $m \in \N$, there is a subsequence $(u_{m_j})_{j \in \N}$ and $u \in Y$ such that $\|u_{m_j} - u\|_Y \to 0$ as $j \to \infty$.
  \end{enumerate}
\end{defn}

The main tool for studying such embeddings will be Sobolev-type inequalities.

\section{Embeddings into \texorpdfstring{$\Ell^q(\Omega)$}{L\textasciicircum q(Omega)} Spaces}

Let us assume $p \in [1,n)$ and ask the question for which $q \in [1,\infty)$ there exists a constant $C > 0$ such that
\begin{equation}
  \label{eq:sobolevTypeIneq}
  \|u\|_{\Ell^q(\R^n)} \leq C \, \|\nabla u\|_{\Ell^p(\R^n)} \quad\text{for all } u \in \CC_0^\infty(\R^n). 
\end{equation}

\begin{motiv}
  We first demonstrate that if \eqref{eq:sobolevTypeIneq} holds then $q$ has to have a specific form:
  To this end, let $u \in \CC_0^\infty(\Omega)$ with $u \not\equiv 0$ and for $\lambda > 0$
  $$
  u_\lambda(x) \coloneqq u(\lambda x) \quad\text{for all } x \in \R^n
  $$
  Then the transformation rule implies
  \begin{align*}
    \int_{\R^n} |u_\lambda(x)|^q \d x &= \int_{\R^n} |u(\lambda x)|^q \d x = \int_{\R^n} |u(y)|^q \d y \\
    \int_{\R^n} |\nabla u_\lambda(x) |^p \d x &= \lambda^p \int_{\R^n} |(\nabla u)(\lambda x)|^p \d x = \frac{\lambda^p}{\lambda^n} \int_{\R^n} |\nabla u(y)|^p \d y
  \end{align*}
  If \eqref{eq:sobolevTypeIneq} holds for any $\lambda > 0$, then 
  $$
  \lambda^{\frac{n}{q}} \| u\|_{\Ell^q(\R^n)}
  = \| u_\lambda\|_{\Ell^q(\R^n)}
  \leq C \, \|\nabla u_\lambda\|_{\Ell^p(\R^n)}
  = C \, \lambda^{1 - \frac{n}{p}} \| \nabla u \|_{\Ell^p(\R^n)}
  $$
  and therefore
  \begin{equation}
    \label{eq:sobolevConjMotiv}
    \|u \|_{\Ell^q(\R^n)} \leq C \, \lambda^{1 - \frac{n}{p} + \frac{n}{q}} \|\nabla u\|_{\Ell^p(\R^n)}.
  \end{equation}
  But if $1 - \frac{n}{p} + \frac{n}{q} \neq 0$, then sending $\lambda$ to either $0$ or $\infty$ in \eqref{eq:sobolevConjMotiv} yields $u \equiv 1$, a contradiction.

  Hence, if \eqref{eq:sobolevTypeIneq} is satisfied, then necessarily we must have $\frac{1}{q} = \frac{1}{p} - \frac{1}{n}$ and $q = \frac{n p }{n - p}$.
  This motivates the following definition.
\end{motiv}

\begin{defn}
  If $p \in [1,n)$, then $p^* \coloneqq \frac{np}{n - p}$ is the \emph{Sobolev conjugate} of $p$.
  In particular, $\frac{1}{p^*} = \frac{1}{p} - \frac{1}{n}$.
\end{defn}

We next prove that \eqref{eq:sobolevTypeIneq} is in fact true for $q = p^*$.

\begin{thm}[Gagliardo-Nirenberg-Sobolev inequality]
  \label{thm:gagliardoNirenberg}
  Assume that $p \in [1,n)$.
  Then there exists $C > 0$, depending only on $p$ and $n$, such that
  \begin{equation}
    \label{eq:gagliardoNirenbergSobolevIneq}
    \| u\|_{\Ell^{p^*}(\R^n)} \leq C \, \|\nabla u\|_{\Ell^p(\R^n)}, \quad\text{for all } u \in \CC_0^1(\R^n).
  \end{equation}
  Moreover, we have $\WW^{1,p}(\R^n) \subset \Ell^{p^*}(\R^n)$ and \eqref{eq:gagliardoNirenbergSobolevIneq} holds for all $u \in \WW^{1,p}(\R^n)$
\end{thm}

\begin{rem}
  We need that $u$ in \eqref{eq:gagliardoNirenbergSobolevIneq} has compact support, as $u \equiv 1$ shows, but $C$ does not depend on the size of the support.
\end{rem}

\begin{proof}[Proof of Theorem \ref{thm:gagliardoNirenberg}]
  \begin{enumerate}[i)]
    \item Let $u \in \CC_0^1(\R^n)$ and $p = 1$, whence $p^* = \frac{n}{n - 1}$.
      Since $\supp(u)$ is compact, we have for all $x \in \R^n$ and $i \in \{1,\dots,n\}$
      \begin{align*}
        u(x) &= \int_{-\infty}^{x_i} u_{x_i}(x_1,\dots,x_{i - 1}, y_i, x_{i + 1}, \dots, x_n) \d y_i \\
        \intertext{and}
        |u(x)| &\leq \int_{-\infty}^\infty |\nabla u(x_1,\dots,x_{i - 1}, y_i, x_{i + 1}, \dots,x_n) \d y_i. \\
        \intertext{Hence,}
        |u(x)|^{\frac{n}{n - 1}} &\leq \prod_{i = 1}^n \left( \int_{-\infty}^\infty |\nabla u(x_1,\dots,x_{i - 1}, y_i, x_{i + 1}, \dots, x_n) | \d y_i \right)^{\frac{1}{n - 1}}.
      \end{align*}
      Integrating this with respect to $x_1$ and using the general Hölder inequality (\ref{sec:lpBasics} viii) with $p_k = n - 1, k = 1,\dots,n - 1$)
      \begin{align*}
        \int_{-\infty}^\infty |u|^{\frac{n}{n - 1}}(x) \d x_1
        &\leq \int_{-\infty}^\infty \prod_{i = 1}^n \left( \int_{-\infty}^\infty |\nabla u| \d y_i \right)^{\frac{1}{n - 1}} \d x_1 \\
        &= \left( \int_{-\infty}^\infty |\nabla u| \d y_1 \right)^{\frac{1}{n - 1}} \, \int_{-\infty}^\infty \prod_{i = 2}^n \left( \int_{-\infty}^\infty |\nabla u| \d y_i \right)^{\frac{1}{n - 1}} \d x_1 \\
        &\leq \left( \int_{-\infty}^\infty |\nabla u| \d y_1 \right)^{\frac{n}{n - 1}} \, \prod_{i = 2}^n \left( \int_{-\infty}^\infty \int_{-\infty}^\infty |\nabla u| \d y_i \d x_1 \right)^{\frac{1}{n - 1}}.
      \end{align*}
      Integrating with respect to $x_2$ and using again the general Hölder inequality, we get
      \begin{align*}
        \int_{-\infty}^\infty &\int_{-\infty}^\infty |u(x)|^{\frac{n}{n - 1}} \d x_1 \d x_2 \\
        &\leq \left( \int_{-\infty}^\infty \int_{-\infty}^\infty |\nabla u| \d y_2 \d x_1 \right)^{\frac{1}{n - 1}} \\
        &\quad\quad \cdot \left[ \int_{-\infty}^\infty \left( \int_{-\infty}^\infty |\nabla u| \d y_1 \right)^{\frac{1}{n - 1}} \prod_{i = 3}^n \left( \int_{-\infty}^\infty \int_{-\infty}^\infty  |\nabla u| \d x_1 \d y_i \right)^{\frac{1}{n - 1}} \d x_2 \right] \\
        &\leq \left( \int_{-\infty}^\infty \int_{-\infty}^\infty |\nabla u| \d y_2 \d x_1 \right)^{\frac{1}{n - 1}} \\
        &\quad\quad \cdot \left( \int_{-\infty}^\infty \int_{-\infty}^\infty | \nabla u| \d y_1 \d x_2 \right)^{\frac{1}{n - 1}} \prod_{i = 3}^n \left( \int_{-\infty}^\infty \int_{-\infty}^\infty \int_{-\infty}^\infty |\nabla u| \d x_1 \d x_2 \d y_i  \right)^{\frac{1}{n - 1}}
      \end{align*}
      Continuing like this and integrating with respect to $x_3,\dots,x_n$ we finally have
      \begin{align}
        \label{eq:gagliardoNirenbergStep1}
        \int_{\R^n} |u(x)|^{\frac{n}{n - 1}} \d x
        &\leq \prod_{i = 1}^n \left( \int_{-\infty}^\infty \cdots \int_{-\infty}^\infty |\nabla u| \d x_1 \dots \d y_i  \dots \d x_n \right)^{\frac{1}{n - 1}} \\
        &= \left( \int_{\R^n} |\nabla u(x)| \d x \right)^{\frac{n}{n - 1}}\nonumber
      \end{align}
      which establishes \eqref{eq:gagliardoNirenbergSobolevIneq} for $p = 1$ with $C = 1$.

    \item Let $p \in (1,n)$ and $u \in \CC_0^1(\R^n)$.
      For $\gamma > 1$, $v \coloneqq |u|^\gamma \in \CC_0^1(\R^n)$ and by \eqref{eq:gagliardoNirenbergStep1} applied to $v$ we have by the Hölder inequality
      \begin{align}
        \label{eq:gagliardoNierenberg2}
        \left(\int_{\R^n} |u|^{\frac{\gamma n}{n - 1}} \right)^{\frac{n - 1}{n}}
        &\leq \int_{\R^n} \Big| \nabla (|u|^\gamma) \Big| \d x 
        = \int_{\R^n} \gamma | u|^{\gamma - 1} | \nabla u| \d x \\ 
        &\leq \gamma \left(\int_{R^n} |u|^{(\gamma - 1) \frac{p}{p - 1}} \d x \right)^{\frac{p - 1}{p}} \left( \int_{\R^n} | \nabla u|^p \d x \right)^{\frac{1}{p}}. \nonumber
      \end{align}
      Choosing $\gamma \coloneqq \frac{p (n - 1)}{n - p} > 1$, we have $\frac{\gamma n}{n - 1} = (\gamma - 1)\frac{p}{p - 1}$ and hence
      $$
      \frac{\gamma n}{n - 1} = \frac{(np - p) + (p - n)}{n - p} \cdot \frac{p }{p - 1} = \frac{n p }{n - p} = p^*.
      $$
      Thus, in view of $\frac{n - 1}{n} - \frac{p - 1}{p} = \frac{1}{p} - \frac{1}{n} = \frac{1}{p^*}$, \eqref{eq:gagliardoNierenberg2} becomes \eqref{eq:gagliardoNirenbergSobolevIneq} for $C = \gamma$.

    \item As $\WW^{1,p}(\R^n) = \WW_0^{1,p}(\R^n)$ by Corollary \ref{cor:W0andWcoincide} for $u \in \WW^{1,p}(\R^n)$ there is $(u_m)_{m \in \N} \subset \C_0^\infty(\R^n)$ such that $u_m \to u$ in $\WW^{1,p}(\R^n)$.
      Hence, $\nabla u_m \to \nabla u$ in $\Ell^p(\R^n)$ and, as \eqref{eq:gagliardoNirenbergSobolevIneq} holds for any $u_m$, in the limit $l \to \infty$ we conclude by Fatou's lemma that $u$ satisfies \eqref{eq:gagliardoNirenbergSobolevIneq} since $u_{m_l} \to u$ a.e. in $\R^n$.
      Als \eqref{eq:gagliardoNirenbergSobolevIneq} therefore holds for all $u \in \WW^{1,p}(\R^n)$, $\WW^{1,p}(\R^n)$ is continuously embedded into $\Ell^{p^*}(\R^n)$. \qedhere
  \end{enumerate}
\end{proof}

Using the previous result, we get a corresponding result on bounded domains.

\begin{thm}[Embeddings for $\WW^{1,p}(\Omega), p < n$]
  Let $\Omega \subset \R^n$ be a bounded domain with $\partial\Omega \in \CC^1$ and $p \in [1,n)$.
    Then $\WW^{1,p}(\Omega)$ is continuously embedded into $\Ell^q$ for all $q \in [1,p^*]$.
    More precisely, for any $q \in [1,p^*]$ there is $C > 0$ depending on $p,q,n$ and $\Omega$ such that
    \begin{equation}
      \|u\|_{\Ell^q(\Omega)} \leq C \, \|\nabla u\|_{\WW^{1,p}(\Omega)} \quad\text{for all } u \in \WW^{1,p}(\Omega).
    \end{equation}
\end{thm}

\begin{proof}
  Since $\partial \Omega \in \CC^1$, by Theorem \ref{thm:extension} there exists $v \coloneqq E(u) \in \WW^{1,p}(\R^n)$ such that $v = u$ a.e. in $\Omega$ and $\|v\|_{\WW^{1,p}(\R^n)} \leq C_1 \, \|u\|_{\WW^{1,p}(\Omega)}$ with $C_1 > 0$ depinding only on $p$ and $\Omega$.
  Then by \ref{thm:gagliardoNirenberg} we have $v \in \Ell^{p^*}(\R^n)$ and
  $$
  \| u\|_{\Ell^{p^*}(\Omega)} = \|v\|_{\Ell^{p^*}(\Omega)} \leq \|v\|_{\Ell^{p^*}(\R^n)} \leq C_2 \, \|\nabla v\|_{\Ell^p(\R^n)} \leq C_1 C_2 \, \|u\|_{\WW^{1,p}(\Omega)}
  $$
  where $C_2$ is the constant from \eqref{eq:gagliardoNirenbergSobolevIneq} (depending on $p$ and $n$).
  This shows the claim for $q = p^*$.

  As $\Omega$ is bounded, Hölder's inequality implies that $u \in \Ell^q(\Omega)$ for $q \in [1,p^*)$ with
  \begin{align}
    \label{eq:embeddingpltn}
    \|u\|_{\Ell^q(\Omega)}
    & \leq \left[ \left( \int_\Omega 1^{\frac{p^*}{p^* - q}} \d x \right)^{\frac{p^* - q}{p^*}} \cdot \left( \int_\Omega u^{q  \frac{p^*}{q}} \right)^{\frac{q}{p^*}} \right]^{\frac{1}{q}} \\
    &= |\Omega|^{\frac{p^* - q}{p^* q}} \|u\|_{\Ell^{p^*}(\Omega)}
    \leq |\Omega|^{\frac{p^* - q}{p^* q}} C_1 C_2 \, \|u\|_{\WW^{1,p}(\Omega)} \nonumber. \qedhere
  \end{align}
\end{proof}

For $\WW_0^{1,p}(\Omega)$, we get these embeddings without assumptions on $\partial\Omega$.

\begin{thm}
  Let $\Omega \subset \R^n$ be a bounded domain and $p \in [1,n)$.
  Then for any $q \in [1,p^*]$ there exists $C > 0$ depending only on $p,q,n$ and $\Omega$ such that 
  \begin{equation}
    \|u\|_{\Ell^q(\Omega)} \leq C \, \|\nabla u\|_{\Ell^p(\Omega)} \quad\text{for all } u \in \WW_0^{1,p}(\Omega).
  \end{equation}
  In particular, $\WW_0^{1,p}(\Omega)$ is continuously embedded into $\Ell^q(\Omega)$.
\end{thm}

\section{Embeddings into Hölder Spaces}

\section{General Embeddings and Sobolev Inequalities}
