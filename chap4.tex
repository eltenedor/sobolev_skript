\chapter{Approximation by Smooth Functions}

As it is often complicated to use the definition of weak derivatives for proving properties of Sobolev spaces, we aim to approximate functions in Sobolev spaces by smooth functions.

\section{Interior Approximation}

We prove that mollification from \ref{sec:mollifier} provides approximating functions in $\WW_{\loc}^{k,p}(\Omega)$.

\begin{thm}\label{thm:interiorApproximation}
  Let $\Omega \subseteq \R^n$ be open, $k \in \N$, $p \in [1,\infty)$, and $u \in \WW^{k,p}(\Omega)$.
    Then the following statements hold:
    \begin{enumerate}[a)]
      \item $u_\varepsilon \in \CC^\infty(\Omega)$ and $\DD^\alpha(u_\varepsilon)(x) = (\DD^\alpha u)_\varepsilon(x)$ for all $ x \in \Omega_\varepsilon$ and all $\alpha \in \N_0^n$ with $|\alpha| \leq k$.
      \item $u_\varepsilon \to u$ in $\WW_{\loc}^{k,p}(\Omega)$ as $\varepsilon \downarrow 0$.
    \end{enumerate}
\end{thm}

\begin{proof}
  \begin{enumerate}[a)]
    \item By Theorem \ref{thm:mollification}, we have $u_\varepsilon \in \CC^\infty(\Omega)$ and for $|\alpha| \leq k$
  $$
    \DD^\alpha u_\varepsilon(x) = \int_\Omega \DD_x^\alpha \eta_\varepsilon(x - y) u(y) \d y, \quad x \in \Omega,
  $$
    see proof of Theorem \ref{thm:mollification} a), d).
    For fixed $x \in \Omega_\varepsilon$, $\phi(y) \coloneqq \eta_\varepsilon(x - y)$ satisfies $\phi \in \CC_0^\infty(\Omega)$ since $\supp \phi = \overline{\BB_\varepsilon(x)}$ and therefore
    \begin{align*}
      \DD^\alpha (u_\varepsilon)(x) 
      &= (-1)^{|\alpha|} \int_\Omega \DD_y^\alpha (\eta_\varepsilon (x - y))\, u(y) \d y
      = (-1)^{|\alpha|} \int_\Omega \DD^\alpha \phi(y) \, u(y) \d y \\
      &\overset{\eqref{eq:weakPartialDerivativeDef}}{=} (-1)^{|\alpha| + |\alpha|} \int_\Omega \phi(y) \, \DD^\alpha u(y) \d y
      = \int_\Omega \eta_\varepsilon(x - y)\, \DD^\alpha u(y) \d y.
      = (\DD^\alpha u)_\varepsilon(x).
    \end{align*}
    Since $x \in \Omega_\varepsilon$ was arbitrary, this proves a).

  \item In view of a) and Theorem \ref{thm:mollifier} d), for fixed $V \Subset \Omega$ we have $\DD^\alpha u_\varepsilon = \eta_\varepsilon \ast \DD^\alpha u$ in $V$ for $\varepsilon \in (0,\varepsilon_0)$, as $V \subset \Omega_\varepsilon$ for $\varepsilon$ small enough so that $\DD^\alpha u_\varepsilon \to \DD^\alpha u$ in $\Ell^p(V)$ as $\varepsilon \downarrow 0$ for any $\alpha \in \N_0^n, |\alpha| \leq k$.
     Then
     \begin{align*}
     \|u_\varepsilon - u\|^p_{\WW^{k,p}(V)} = \sum_{|\alpha| \leq k} \|\DD^\alpha u_\varepsilon - \DD^\alpha u\|_{\Ell^p(V)}^p \to 0 \quad\text{as } \varepsilon \downarrow 0. &\qedhere
   \end{align*}
 \end{enumerate}
\end{proof}

\section{Approximation by Smooth Functions}

In order to show that for any $u \in \WW^{k,p}(\Omega)$ there is $(u_m)_{m \in \N} \subset \CC^\infty(\Omega) \cap \WW^{k,p}(\Omega)$ such that $u_m \to u$ in $\WW^{k,p}(\Omega)$ (and not only in $\WW^{k,p}_{\loc}(\Omega)$), we need the following lemmas to construct a partition of unity.

\begin{lem}
  \label{lem:cutoff}
  Let $\Omega \subset \R^n$ be open and $K \subset \Omega$ compact.
  If $\dist(K, \partial \Omega) \geq \delta > 0$, then there exists a cutoff\hyp{}function $\tau \in \CC_0^\infty(\Omega)$ w.r.t. $K, \Omega$ with $0 \leq \tau \leq 1$, $\tau = 1$ in $K$, and 
  $$
|\DD^\alpha \tau(x) | \leq c \delta^{-k} \quad\text{for all } x \in \Omega \setminus K, k \in \N , |\alpha| = k,
  $$ 
  where $c > 0$ depends on $k$ and $n$ and not on $\Omega$ or $K$.
\end{lem}

\begin{proof}
  We may choose $\delta > 0$ since $K$ is compact.
  Hence, 
  $$
  \tilde K \coloneqq \overline{\bigcup_{x \in K} \BB_{\frac{\delta}{2}}(x)}
  $$
  is compact with $\dist(\partial \tilde K, \partial K) = \frac{\delta}{2} \leq \dist(\delta \tilde K, \partial \Omega)$.

  As $\chi_{\tilde K} \in \Ell^1(\Omega)$ with $\supp \chi_{\tilde K} = \tilde K \Subset \Omega$, we have that $\tau \coloneqq \eta_{\frac{\delta}{4}} \ast \chi_{\tilde K}$ satisfies $\tau \in \CC_0^\infty(\Omega)$, $0 \leq \tau \leq 1$, and $\tau = 1$ in $K$ by Theorem \ref{thm:mollifier}, as
  $$
  \tau(x) 
  = \int_{\BB_{\frac{\delta}{4}}(x)}\eta_{\frac{\delta}{4}}(x - y) \underbrace{\chi_{\tilde K}(y)}_{= 1} \d y 
  = 1 \quad\text{for all }x \in K
  $$
  since $\BB_{\frac{\delta}{4}}(x) \subset \tilde K$.
  Moreover, for $|\alpha| = k$
  $$
  \DD^\alpha \eta_{\frac{\delta}{4}}(x) 
  = \left( \frac{4}{\delta}\right)^n \DD^\alpha \left[ \eta \left( \frac{4 }{\delta} x \right) \right] 
  =  \left( \frac{4}{\delta}\right)^{n+k} (\DD^\alpha \eta)\left( \frac{4}{\delta} x \right).
  $$
  Hence, for $x \in \Omega \setminus K$ we have 
  $$
    |\DD^\alpha \tau (x) | 
    \leq \int_{\BB_{\frac{\delta}{4}}(x)}\, \left(\frac{4}{\delta}\right)^{n + k}\, \|\DD^\alpha \eta \|_{\Ell^\infty(\R^n)} \,\chi_{\tilde K}(y) \d y
    \leq \tilde c(n,k) \, \delta^{-n - k} \, |\BB_{\frac{\delta}{4}}(x)| \leq c(n,k)\, \delta^{-k}
 $$
 which concludes the proof.
\end{proof}

\begin{lem}[Partition of unity]
  \label{lem:partitionOfUnity}
  Let $K \subset \R^n$ be compact and $\{\Omega_k\}_{k = 1,\dots,N}$ be an open covering of $K$.
  Then, there exist $\psi_k, k=1,\dots,N$, called \emph{partition of unity} such that $\psi_k \in \CC_0^\infty(\Omega_k)$, $0 \leq \psi_k \leq 1$ in $\Omega_k$, and $\sum_{k = 1}^N \psi_k(x) = 1$ for all $x \in K$.
\end{lem}

\begin{proof}
  For any $x \in K$ there is $r = r(x) > 0$ and $1\leq k \leq N$ such that $\BB_{x, k} \coloneqq \BB_r(x) \Subset \Omega_k$.
  Hence, $$\{B_{x,k}\}_{\substack{x \in K \cap \Omega_k, \\ k = 1,\dots,N}}$$ is an open covering of $K$ and has a finite subset still covering $K$, called $$\{B_i^k\}_{\substack{i = 1,\dots,N_k \\ k = 1,\dots,N}}.$$ Then
  $$
  K_k \coloneqq \overline{\bigcup_{i = 1}^{N_k} B_i^k}
  $$
  satisfies $K_k \Subset \Omega_k$ and $\bigcup_{k = 1}^N K_k \supset K$.
  Let $\tilde \psi_k$ denote the cutoff\hyp{}function w.r.t $K_k, \Omega_k$.
  Hence, $\tilde \psi_k \in \CC_0^\infty(\Omega_k)$ satisfies $0 \leq \tilde \psi_k \leq 1$ and
  $$
  \psi(x) \coloneqq \sum_{k = 1}^N \tilde \psi_k(x) \geq 1 \quad\text{for all } x \in K.
  $$
  Furthermore, we have
  $$
  K \Subset \Omega \coloneqq \bigcup_{k = 1}^N \supp(\tilde \psi_k)
  $$
  and there is an open set $\Omega_0$ such that $K \subset \Omega_0 \Subset \Omega$.
  Let $\tau$ be a cutoff\hyp{}function w.r.t $K, \Omega_0$ and
  $$
  \psi_k(x) \coloneqq \begin{cases} \frac{\tilde \psi_k(x) \tau(x)}{\psi(x)} &\quad\text{if } x \in \Omega_0, \\ 0 &\quad \text{if } x \not\in \Omega_0. \end{cases}
  $$
  Then $\psi_1,\dots,\psi_N$ have the claimed properties.
\end{proof}

Now we prove the announced result that $\CC^\infty(\Omega) \cap \WW^{k,p}(\Omega)$ is dense in $\WW^{k,p}(\Omega)$ without assuming any smoothness of $\partial \Omega$.

\begin{thm}[Meyers and Serrin]
  \label{thm:meyersSerrin}
  Let $\Omega \subset \R^n$ be open, $k \in \N$, and $p \in [1,\infty)$.
    Then $\CC^\infty(\Omega) \cap \WW^{k,p}(\Omega)$ is dense in $\WW^{k,p}(\Omega)$, i.e. for any $u \in \WW^{k,p}(\Omega)$ there exists $(u_m)_{m \in \N} \subset \CC^\infty(\Omega) \cap \WW^{k,p}(\Omega)$ such that $u_m \to u$ in $\WW^{k,p}(\Omega)$ as $m \to \infty$.
\end{thm}

\begin{proof}
  \begin{enumerate}[i)]
  \item With
    $$
    U_i \coloneqq \{ x \in \Omega \colon \dist(x, \partial \Omega) > \frac{1}{i} \text{ and } |x| < i\}, \quad i \in \N,
    $$
    we have $\bigcup_{i = 1}^\infty U_i = \Omega$ and $U_i \subset U_{i + 1}$.
    Moreover, 
    $$V_i \coloneqq U_{i + 4} \setminus \overline{U_{i + 1}}, i \in \N,\quad\text{and}\quad V_0\coloneqq U_4$$
    are all open with $V_i \Subset \Omega$ for all $i \in \N_0$ and  $\Omega = \bigcup_{i = 0}^\infty V_i$.
    Defining further 
    $$
    W_i \coloneqq \overline{U_{i + 3}} \setminus U_{i + 2}, i \in \N, \quad\text{and}\quad W_0 \coloneqq \overline U_3,
    $$
    all $W_i \subset V_i$ are compact and we have $\Omega = \bigcup_{i = 0}^\infty W_i$.
      Let $\psi_i \in \CC_0^\infty(V_i)$ denote a cutoff\hyp{}function w.r.t. $W_i, V_i$ with $0 \leq \psi_i \leq 1$ and $\psi_i = 1$ in $W_i$ for $i \in \N_0$.
    Since for all $j \geq i + 2$
    $$W_i \cap V_j 
    = \left(\overline{U_{i + 3}} \cap \overline{U_{j + 1}}^{\mathrm c}  \right)\cap \left( U_{i + 2}^{\mathrm c} \cap  U_{j + 4}   \right) 
    =  \emptyset,$$ 
    and for all $j \geq i + 3$, $V_i \cap V_j = \emptyset$, for any $x \in \Omega$ we have
    $$
    \sigma(x) \coloneqq \sum_{i = 0}^\infty \psi_i(x) > 0
    $$
    and only finitely many of the $\psi_i(x)$ are non-zero.
    Hence, $\{\xi_i\}_{i = 0}^\infty$, defined by
    $$\xi_i(x) \coloneqq \frac{\psi_i(x)}{\sigma(x)}, \quad x \in \Omega,$$
      is a \emph{partition of unity subordinate to} $\{V_i\}_{i = 0}^\infty$, i.e. $\xi_i \in \CC_0^\infty(V_i)$, $0 \leq \xi_i \leq 1$, and $\sum_{i = 0}^\infty \xi_i = 1$ in $\Omega$ and for any $K \Subset \Omega$, $\xi_i|_K \not\equiv 0$ only for finitely many $i$.
  \item Let $u \in \WW^{k,p}(\Omega)$ be arbitrary.
    Then by Proposition \ref{prop:sobolevProperties} d) and i) we have $\xi_i u \in \WW^{k,p}(\Omega)$ and $\supp(\xi_i u) \subset V_i$ for all $i \in \N_0$.
    We fix $\delta > 0$.
    Then for any $i \in \N_0$ we define
    $$
    Z_i \coloneqq U_{i + 5} \setminus \overline{U_i} \supset V_i, i \in \N, \quad\text{and}\quad Z_0 \coloneqq U_5 \supset V_0.
    $$
    In view of Theorem \ref{thm:interiorApproximation}, there is $\varepsilon_i > 0$ small enough such that $u_i \coloneqq \eta_{\varepsilon_i} \ast (\xi_i u)$ satisfies $u_i \in \CC_0^\infty(Z_i)$ and
    \begin{equation}
      \|u_i - \xi_i u \|_{W^{k,p}(\Omega)} = \|u_i - \xi_i u\|_{W^{k,p}(Z_i)} \leq \frac{\delta}{2^{i + 1}}
    \end{equation}
    for $i \in \N_0$ as $u_i - \xi_i u \equiv 0$ in $\Omega \setminus Z_i$.
    Define
    $$
    v(x) \coloneqq \sum_{i = 0}^\infty u_i(x), \quad x \in \Omega.
    $$
    Then for any open set $V \Subset \Omega$ only finitely many $u_i$ satisfy $u_i|_V \not\equiv 0$.
    Since $u = \sum_{i = 0}^\infty \xi_i u$, we obtain $v \in \CC^\infty(\Omega)$ and
    $$
    \|v - u\|_{\WW^{k,p}(V)}
    \leq \sum_{i = 0}^\infty \|u_i - \xi_i u\|_{\WW^{k,p}(V)}
    \overset{(4.1)}{\leq} \delta \sum_{i = 0}^\infty \frac{1}{2^{i + 1}} 
    = \delta \quad\text{for all } V \Subset \Omega.
    $$
    Since $U_i \subset U_{i + 1}$ for all $i \in \N$, $U_i \Subset \Omega$, and $\Omega = \bigcup_{i = 1}^\infty U_i$, we conclude by the monotone convergence theorem
    $$
    \|v - u\|_{\WW^{k,p}(\Omega)}^p
    = \sum_{|\alpha| \leq k} \|\DD^\alpha (v - u) \|_{\Ell^p(\Omega)}^p
    = \lim_{i \to \infty} \sum_{|\alpha| \leq k} \|\DD^\alpha(v - u) \|_{\Ell^p(U_i)}^p
    \leq \delta^p.
    $$
    As $\delta > 0$ was arbitrary, the claim is proved. \qedhere
\end{enumerate}
\end{proof}

\begin{rem}
  Historically, there were two definitions of Sobolev spaces.
  $\WW^{k,p}(\Omega)$ was defined as in Definition \ref{defn:weakPartialDerivative} while $\HH^{k,p}(\Omega)$ was defined as the closure of $\CC^\infty(\Omega) \cap \WW^{k,p}(\Omega)$ w.r.t $\|\cdot\|_{W^{k,p}(\Omega)}$.
  Obviously $\HH^{k,p}(\Omega) \subseteq \WW^{k,p}(\Omega)$ but only after Meyers and Serrin in 1964 \cite{meyers1964} it was clear that $\HH^{k,p}(\Omega) \supseteq \WW^{k,p}(\Omega)$ without assuming any smoothness condition of $\partial \Omega$.
\end{rem}

We can now prove the chain rule for $\WW^{1,p}(\Omega)$ functions.

\begin{prop}
  Let $\Omega$ be open $p \in [1,\infty)$ and $f \in \CC^1(\R)$ such that $|f'| \leq M$ on $\R$ for some $M > 0$.
  Assume further that $f(0) = 0$ or $|\Omega| < \infty$ is satisfied.
  Then for any $u \in \WW^{1,p}(\Omega)$ and $u_m \to u$ a.e. in $\Omega$ we have $f(u)$ in $\WW^{1,p}(\Omega)$ with $$\nabla f(u) = f'(u) \nabla u.$$
\end{prop}

\begin{proof}
  As $f'$ is continuous and bounded and $u$ is measurable, we have $f'(u)\in \Ell^\infty(\Omega)$ and $f'(u) \nabla u \in \Ell^p(\Omega)$.
  In view of 
  $$
  |f(x)| \leq |f(0)| + M|x| \quad\text{for all } x \in \R,
  $$
  the assumption $f(0)= 0$ or $|\Omega| < \infty$ implies $f(u) \in \Ell^p(\Omega)$.
  By Theorem \ref{thm:meyersSerrin}, there exists $(u_m)_{m \in \N} \subset \CC^\infty(\Omega) \cap \WW^{1,p}(\Omega)$ such that $u_m \to u$ in $\WW^{1,p}(\Omega)$. 
  Hence, $u_m \to u$ and $(u_m)_{x_i} \to u_{x_i}$ in $\Ell^p(\Omega)$ for all $i \in \{1,\dots,n\}$.
  We fix $i \in \{ 1, \dots, n\}$ and $\varphi \in \CC_0^\infty(\Omega)$.
  In view of $f(u_m) \in \CC^1(\Omega)$, we deduce from \eqref{eq:partialIntegrationGeneral}
  \begin{equation}
    \label{eq:partialIntChainRule}
    \int_\Omega f(u_m) \, \varphi_{x_i} \d x = -\int_\Omega f'(u_m) (u_m)_{x_i} \,  \varphi \d x \quad\text{for all } m \in \N,
  \end{equation}
  by the classical chain rule.
  On the one hand, for $q \in [1,\infty]$ with $\frac{1}{p} + \frac{1}{q} = 1$  we have
  \begin{align*}
    \Big| \int_\Omega (f(u_m) - f(u))\, \varphi_{x_i} \d x \Big|
    &\leq M \int_{\Omega} |u_m - u|\, |\varphi_{x_i}| \d x \\
    &\leq M\, \|u_m - u\|_{\Ell^p(\Omega)} \|\varphi_{x_i}\|_{\Ell^q(\Omega)} 
    \to 0 \quad\text{as } m \to \infty
  \end{align*}
  since $f$ is Lipschitz.
  On the other hand, 
  $$
  |f'(u_m) - f'(u)|\, |u_{x_i}|\, |\varphi| \leq 2M\, |u_{x_i}| \|\varphi\|_{\Ell^\infty(\Omega)} \in \Ell^1(\supp(\varphi))
  $$
  as $\supp(\varphi)$ is bounded and thus
  \begin{align*}
    &\Big |\int_\Omega (f'(u_m) (u_m)_{x_i} - f'(u)u_{x_i})\, \varphi \d x \Big| \\
    &\quad\leq \int_\Omega |f'(u_m)|\, |(u_m)_{x_i} - u_{x_i}|\,| \varphi | \d x 
    + \int_\Omega |f'(u_m) - f'(u)|\, |u_{x_i}|\, |\varphi| \d x \\
    &\quad\leq M\, \|(u_m)_{x_i} - u_{x_i}\|_{\Ell^p(\Omega)} \|\varphi\|_{\Ell^q(\Omega)}
    + \int_{\supp(\varphi)} |f'(u_m) - f'(u)|\, |u_{x_i}|\, |\varphi| \d x .
    \\&\quad \to 0 \quad\text{as } m \to \infty
  \end{align*}
   by the dominated convergence theorem.

   Hence, letting $m \to \infty$ in \eqref{eq:partialIntChainRule}  we conclude that $(f(u))_{x_i} = f'(u)u_{x_i}$ in the weak sense.
\end{proof}

Unlike for classical derivatives, now $u \in \WW^{1,p}(\Omega)$ implies $|u| \in \WW^{1,p}(\Omega)$.

\begin{cor}
  Let $\Omega \subset \R^n$ be open, $p \in [1,\infty)$, and $u \in \WW^{1,p}(\Omega)$.
  Define 
  $$u_+(x) \coloneqq \max\{u(x),0\} \quad\text{and}\quad u_-(x) = \max\{-u(x),0\}.$$
  Then $u_+, u_-,|u| \in \WW^{1,p}(\Omega)$ with
    $\nabla u_+(x) = \nabla u(x) \chi_{\{ u > 0\}}(x)$, $\nabla u_-(x) = \nabla u(x) \chi_{\{ u < 0\}}(x)$, and $\nabla|u|(x) = \nabla u(x) \left( \chi_{\{ u > 0\} } (x) - \chi_{\{u < 0\}} (x) \right)$.
\end{cor}

\begin{proof}
  Exercise.
\end{proof}

\section{Approximation by \texorpdfstring{$\CC^\infty(\overline{\Omega})$}{C\textasciicircum infty(Omega) }-Functions}

We now ask the question whether any $u \in \WW^{k,p}(\Omega)$ can also be approximated by functions $u_m \in \CC^\infty(\overline\Omega)$ instead of $u_m \in \CC^\infty(\Omega)$.
The following example shows that this is not true for all open $\Omega \subset \R^n$.

\begin{ex}
  Let $\Omega = \{(x,y) \in \R^2 \colon 0 < |x| < 1, 0 < y < 1\}$ and $p \in [1,\infty)$.
  Then $u \colon \Omega \to \R$ defined by
  $$
  u(x,y) \coloneqq \begin{cases} 1 \quad\text{if } x > 0,\\ 0 \quad\text{if } x < 0 \end{cases}
  $$
  belongs to $\WW^{1,p}(\Omega)$, but for $\varepsilon > 0$ sufficiently small there is no $\varphi \in \CC^1(\overline\Omega)$ such that $\|\varphi - u\|_{\WW^{1,p}(\Omega)} < \varepsilon$.
\end{ex}

The problem with $\Omega$ in the example is that it lies on both sides of the segment $\Gamma = \{(0,y) \colon y \in [0,1]\}$ with $\gamma \subset \partial \Omega$.
The following condition excludes this situation.
Moreover, we assume from now on that $\Omega$ is a domain, i.e. open and connected.

\begin{defn}
  \label{defn:segmentCond}
  Let $\Omega \subset \R^n$ be a domain.
  We say that $\Omega$ satisfies the \emph{segment condition} if for any $x \in \partial\Omega$ there exists a neighborhood $U_x \subset \R^n$ of $x$ and $0 \neq y_x \in \R^n$ such that $z + t y_x \in \Omega$ for any $z \in \overline\Omega \cap U_x$ and any $\varepsilon \in (0,1)$.
\end{defn}

Another condition on $\partial\Omega$ is that it is locally the graph of a $\CC^m$ function.

\begin{defn}
  \label{defn:CmBoundary}
  Let $\Omega \subset \R^n$ be a bounded domain and $m \in \N$.
  We say that $\Omega$ \emph{is of class} $\CC^m$ or simply $\partial\Omega \in \CC^m$ if for any $x^0 \in \partial \Omega$ there exists $r = r(x^0) > 0$ and $\gamma = \gamma_{x^0} \in \CC^m(\R^{n - 1})$ such that upon relabeling and reorienting the coordinate axes if necessary we have
  \begin{align*}
     \Omega \cap \BB_r(x^0) &= \{ x \in \BB_r(x^0) \colon x_n > \gamma(x_1,\dots,x_{n - 1}) \}, \\
     \partial \Omega \cap \BB_r(x^0) &= \{ x \in \BB_r(x^0) \colon x_n = \gamma(x_1,\dots,x_{n - 1})\}.
  \end{align*}
  Furthermore, we say $\partial \Omega \in \CC^\infty$ if $\gamma \in \CC^\infty$ and we say $\partial \Omega$ \emph{is analytic} if $\gamma$ is analytic.
\end{defn}

\begin{rem}
  Let $\Omega$ be a bounded domain with $\partial \Omega \in \CC^1$.
  Then for any $x^0\in \partial\Omega$ there is a unique outward unit vector $\nu(x^0)$, i.e. $|\nu| = 1$, $\nu(x^0) \perp y$ for all $y \in \TT(x^0)$, where $\TT(x^0)$ is the tangential space on $\partial\Omega$ in $x^0$ and $x^0 + t\nu(x^0) \not\in\overline\Omega$ for all $t \in (0,\varepsilon_0)$ for $\varepsilon_0 > 0$ small.

  Indeed, as 
  $$
  \def\arraystretch{0.6}
  \partial\Omega \cap \BB_r(x^0) = \Bigg\{ x \in \BB_r(x^0) \colon x = \left( \begin{array}{c} x_1 \\ \vdots \\ x_{n - 1} \\ \gamma(x_1,\dots,x_{n-1}) \end{array} \right) \Bigg\} 
  $$
  we have that
  $$
  \def\arraystretch{0.6}
  \TT(x) = \Span \Bigg\{ 
    \left(\begin{array}{c} 1 \\ 0 \\ \vdots \\ 0 \\ \gamma_{x_1}(x_1,\dots,x_{n - 1}) \end{array} \right),
    \dots, 
    \left(\begin{array}{c} 0 \\ 0 \\ \vdots \\ 1 \\ \gamma_{x_{n-1}}(x_1,\dots,x_{n - 1}) \end{array} \right)\Bigg\}
  $$
  is $(n-1)$-dimensional so that $\TT(x)^\perp$ is one-dimensional.
  Hence, $\nu(x)$ is uniquely defined and $\nabla \colon \partial\Omega \to \R^n$ is continuous as $\gamma \in \CC^1(\R^{n - 1})$ and 
  $$
  \nu(x) = \frac{1}{\sqrt{1 + |\nabla \gamma|^2}} \, (\gamma_{x_1},\dots,\gamma_{x_{n - 1}}, -1)^\TT.
  $$
  In particular, $\Omega$ satisfies the segment condition with $y_x = -\nu(x)$ and $U_x = \BB_\rho(x)$ with some $\rho \in (0,r)$ small enough (as $\nu$ is continuous). \hfill$\square$
\end{rem}

\begin{prop}\label{prop:continuousTrans}
  Let $p \in [1,\infty)$ and $u \in \Ell^p(\R^n)$. 
    Then the translation is continuous in $\Ell^p(\R^n)$ in the sense that we have (with $h \in \R^n$)
    $$
    \lim_{|h| \to 0} \| u(\,\cdot + h) - u(\,\cdot\,)\|_{\Ell^p(\R^n)} = 0.
    $$
\end{prop}

\begin{proof}
  Given $\delta > 0$, by \ref{sec:mollifier} there is $\varphi \in \CC_0^\infty(\R^n)$ such that
  $$
  \|u - \varphi\|_{\Ell^p(\Omega)}.
  $$
  But then also
  $$
  \|u(\,\cdot + h) - \varphi(\,\cdot + h) \|_{\Ell^p(\Omega)} = \|u - \varphi\|_{\Ell^p(\R^n)} < \delta.
  $$
  Since $\varphi$ has compact support, it is uniformly continuous on $\R^n$.
  Hence, there is $M > 0$ such that
  $$
  |\varphi(x + h) - \varphi(x) < \frac{\delta}{3|\supp(\varphi)|^{\frac{1}{p}}} \quad\text{for all } x \in \R^n\text{, } h \in \BB_M(0).
  $$
  Hence, for $h \in \R^n$ with $|h| < M$ we have
  \begin{align*}
    \|u(\,\cdot + h) - u(\,\cdot\,)\|_{\Ell^p(\R^n)}
    &\leq \|u(\,\cdot + h) - \varphi(\,\cdot + h) \|_{\Ell^p(\R^n)} + \|\varphi(\,\cdot + h) - \varphi(\,\cdot\,) \|_{\Ell^p(\R^n)} +  \|\varphi - u \|_{\Ell^p(\R^n)} \\
    &\leq \frac{2}{3}\delta +  \|\varphi(\,\cdot + h) - \varphi(\,\cdot\,)\|_{\Ell^\infty(\R^n)} |\supp(\varphi)|^{\frac{1}{p}} \\
    &< \delta
  \end{align*}
  and the claim follows.
\end{proof}

\begin{thm}
  \label{thm:CinftyUpToBoundary}
  Let $\Omega \subset \R^n$ be a domain satisfying the segment condition, $K \in \N$, and $p \in [1,\infty)$.
    Then the set 
    $
    \{ \varphi|_\Omega \colon \varphi \in \CC_0^\infty(\R^n) \}
    $
    is dense in $\WW^{k,p}(\Omega)$.
    In particular, if in addition $\Omega \neq \R^n$, then for any $u \in \WW^{k,p}(\Omega)$ there is $(u_m)_{m \in \N} \subset \CC^\infty(\overline \Omega)$ such that $u_m \to u$ in $\WW^{k,p}(\Omega)$.
\end{thm}

\begin{proof}
  We fix $u \in \WW^{k,p}(\Omega)$ and $\delta > 0$.
  \begin{enumerate}[i)]
    \item In a first step, we show that in case $\Omega$ is unbounded there exists $v \in \WW^{k,p}(\Omega)$ with $\supp(v)$ bounded and $\|u - v\|_{\WW^{k,p}(\Omega)} < \delta$.
      By Lemma \ref{lem:cutoff}, there exists $\tau \in \CC_0^\infty(\BB_2(0))$ such that $0 \leq \tau \leq 1$, $\tau \equiv 1$ in $\overline{\BB_1(0)}$ and there is some $M = M(k) > 0$ such that $|\DD^\alpha \tau(x)| \leq M$ for all $x \in \R^n$ and all $|\alpha| \leq k$ (choose $K = \overline{\BB_1(0)}, \Omega= \BB_2(0), \delta = 1$ in Lemma \ref{lem:cutoff}).
      For $\varepsilon \in (0,1)$, we define $\tau_\varepsilon \coloneqq \tau(\varepsilon x), x \in \R^n$.
      Then $\tau_\varepsilon \equiv 1$ in $\overline{\BB_{\frac{1}{\varepsilon}}(0)}$, $\tau_\varepsilon \in \CC_0^\infty(\BB_{\frac{2}{\varepsilon}}(0))$, and
      \begin{equation}
        |\DD^\alpha \tau_\varepsilon(x)| \leq M \varepsilon^{|\alpha|} \leq M \quad\text{for all } |\alpha| \leq k.
      \end{equation}
      Hence, $v_\varepsilon \coloneqq \tau_\varepsilon u$ has bounded support and belongs to $\WW^{k,p}(\Omega)$ by Proposition \ref{prop:sobolevProperties} d).
      It further satisfies for $|\alpha| \leq k$
      $$
      |\DD^\alpha v_\varepsilon(x)| 
      = | \sum_{\beta \leq \alpha} \binom{\alpha}{\beta} \DD^\beta \tau_\varepsilon(x) \DD^{\alpha - \beta} u(x)|
      \leq M \sum_{\beta \leq \alpha} \binom{\alpha}{\beta} | \DD^{\alpha - \beta} u(x)| \quad\text{for all } x \in\Omega
      $$
      so that for all $\tilde\Omega \subset \Omega$ open we have
      $$
        \|v_\varepsilon\|_{\WW^{k,p}(\tilde\Omega)}
        \leq \sum_{|\alpha|\leq k} \|\DD^\alpha v_\varepsilon\|_{\Ell^p(\tilde\Omega)}
        \leq M \left( \sum_{|\alpha| \leq k} \sum_{\beta\leq\alpha} \binom{\alpha}{\beta} \right) \|u\|_{\WW^{k,p}(\tilde\Omega)}
        \leq c(k) M \|u\|_{\WW^{k,p}(\tilde\Omega)}
      $$
      with some constant $c(k) > 0$.
      Hence,
      \begin{align*}
        \|u - v_\varepsilon\|_{\WW^{k,p}(\Omega)}
        &= \|u - v_\varepsilon\|_{\WW^{k,p}(\Omega \setminus \overline{\BB_{\frac{1}{\varepsilon}}(0)})}
        \leq \|u\|_{\WW^{k,p}(\Omega \setminus \overline{B_{\frac{1}{\varepsilon}}(0)})} + \|v_\varepsilon\|_{\WW^{k,p}(\overline{\Omega \setminus \BB_{\frac{1}{\varepsilon}}(0)})} \\
        &\leq (1 + c(k)M) \|u\|_{\WW^{k,p}(\Omega \setminus \overline{\BB_{\frac{1}{\varepsilon}}(0)})} \\
        &\to 0 \quad\text{as } \varepsilon \to 0
      \end{align*}
      in view of $p < \infty$.
      Hence, $\|u - v_\varepsilon\|_{\WW^{k,p}(\Omega)} < \delta$ for $\varepsilon > 0$ small enough and $v = v_\varepsilon$ has bounded support.

    \item In view of i) we may assume w.l.o.g. that $K \coloneqq \supp(u)$ is bounded and hence compact (if necessary, we replace $u$ by $v$).

      For $x \in \partial\Omega$, let $U_x$ in $\R^n$ be the open neighborhood of $x$ and $0 \neq y_x \in \R^n$ like in Definition \ref{defn:segmentCond}.
      Then
      $$
      F \coloneqq K \setminus ( \bigcup_{x \in \partial\Omega} U_x )
      $$
      is compact with $F \subset \Omega$.
      Hence, there is $U_0$ open such that $F \Subset U_0 \Subset \Omega$.
      As $K$ is compact, there exist finitely many of the sets $U_x$ which we call $U_1,\dots,U_N$ such that $K \subset \bigcup_{i = 0}^N U_i$.
      Moreover, we choose $V_i \Subset U_i$ open sets $i = 0,\dots,N,$ such that $K \subset \bigcup_{i = 0}^N V_i$ and $V_i$ is still a neighborhood of $x^i$ belonging to $U_i = U_{x^i}$.
      By Lemma \ref{lem:partitionOfUnity} there is a partition of unity $\psi_0,\dots,\psi_n$ such that $\psi_i \in \CC_0^\infty(V_i)$, $0\leq \psi_i \leq 1$ for $i = 0,\dots,N,$ and $\sum_{i = 0}^N \psi_i(x) = 1$ for all $x \in K$.

      Our aim is to find $\varphi_i \in \CC_0^\infty(\R^n)$ such that with $u_i \coloneqq \psi_i u$ we have
      \begin{equation} \label{eq:diffUPhi}
        \|u_i - \varphi_i\|_{\WW^{k,p}(\Omega)} < \frac{\delta}{N + 1} \quad\text{for all } i \in \{0,\dots,N\}.        
      \end{equation}
      As $\supp(u_0) \Subset V_0 \Subset \Omega$, by Theorems \ref{thm:interiorApproximation} and \ref{thm:mollification} b) there exists $\varphi_0 \in \CC_0^\infty(\R^n)$ such that \eqref{eq:diffUPhi} holds for $i = 0$.
      Next, we fix $i \in \{ 1, \dots, N\}$ and extend $u$ by $0$ outside $\Omega$.
      Let $x^i \in \partial \Omega$ be the point belonging to $U_i$ ($U_i$ is a neighborhood of $x^i$) and 
      $$
      \Gamma \coloneqq \overline{V_i} \cap \partial\Omega.
      $$
      As $\psi_i = 0$ on $\partial \Omega \setminus \Gamma$, $u_i = 0$ on $\R^n \setminus \overline{\Omega}$, and $u_i \in \WW^{k,p}(\Omega)$ by Proposition \ref{prop:sobolevProperties}, we get $u_i \in \WW^{k,p}(\R^n \setminus \Gamma)$.
      Let $y \coloneqq y_{x_i}$ from the segment condition and 
      $$
      \Gamma_t \coloneqq \{ x - ty \colon x \in \Gamma \},
      $$
      where 
      $$
      0 < t < \min\{1, \frac{1}{|y|} \dist(\partial V_i, \partial U_i)\}.
      $$
      By the choice of $t$, we have $\Gamma_t \subset U_i$ and $\Gamma_t \cap \overline{\Omega} = \emptyset$.
      The latter follows from the segment condition: For $z = x - sy$ with $x \in \Gamma$ and $s \in (0,1)$ we have $z + sy = x \in \Gamma \subset \partial\Omega$.
      Hence, $z \not\in\overline{\Omega}$ so that $\Gamma_t \cap \overline\Omega = \emptyset$.
      Define 
      $$
      w_t(x) \coloneqq u_i(x + ty).
      $$
      As $u_i \in \WW^{k,p}(\R^n \setminus \Gamma)$, we have $w_t \in \WW^{k,p}(\R^n\setminus \Gamma_t)$.
      Hence, Proposition \ref{prop:continuousTrans} yields that $\DD^\alpha w_t \to \DD^\alpha u_i$ in $\Ell^p(\Omega)$ as $t \downarrow 0$ for all $|\alpha| \leq k$ (since $\overline\Omega \subset \R^n \setminus \Gamma_t$) and we can choose $t$ small enough such that $\|w_t - u_i \|_{\WW^{k,p}(\Omega)} \leq \frac{\delta}{2 ( N + 1 )}$.

      Moreover, since $u \in \Ell^p(\Omega)$ and $u = 0$ on $\R^n \setminus \Omega$, we have $u \in \Ell^p(\R^n)$, $u_i = \psi_i u \in \Ell^p(\R^n)$, and $w_t \in \Ell^p(\R^n)$.
      Hence, by Theorem \ref{thm:interiorApproximation} $\varphi_i \coloneqq \eta_\varepsilon \ast w_t$ belongs to $\CC_0^\infty(\R^n)$ for $\varepsilon > 0$.
      As $\dist(\Gamma_t , \overline\Omega) > 0$, we may choose $\varepsilon > 0$ small enough such that $\|\varphi_i - w_t \|_{\WW^{k,p}(\Omega)} \neq \frac{\delta}{2(N+1)}$ (as $\Omega \cap \supp(w_t) \Subset \R^n \setminus \Gamma_t$ and $w_t \in \WW^{k,p}(\R^n \setminus \Gamma_t)$ we may apply Theorem \ref{thm:interiorApproximation}).
      Altogether, $\varphi_i \in \CC_0^\infty(\R^n)$ satisfies $\|u_i - \varphi_i\| \leq \frac{\delta}{N + 1}$ and \eqref{eq:diffUPhi} holds for all $i \in \{0,\dots,N\}$.
      As $u = \sum_{i = 1}^n u_i$, the function
      $$\varphi \coloneqq \sum_{i = 0}^N \varphi_i \in \CC_0^\infty(\R^n)$$
      satisfies $\| u - \varphi \|_{\WW^{k,p}(\Omega)} \leq \delta$.
    \item Combining i) and ii), the claim is proved.  \qedhere
  \end{enumerate}
\end{proof}

As a Corollary, we see that $\WW_0^{k,p}(\R^n)$ and $\WW^{k,p}(\R^n)$ coincide.

\begin{cor}
  For $k \in \N$ and $p \in [1,\infty)$, we have $\WW_0^{k,p}(\R^n) = \WW^{k,p}(\R^n)$.
\end{cor}

\begin{proof}
  Given $u \in \WW^{k,p}(\R^n)$ by part i) of the proof of Theorem \ref{thm:CinftyUpToBoundary}, there exists $v \in\WW^{k,p}(\R^n)$ with $K\coloneqq \supp(V)$ compact and $$\|u - v\|_{\WW^{k,p}(\R^n)} \leq \frac{\delta}{2}.$$
  But then there is $V_0$ open such that $K \Subset V_0 \Subset \R^n$ and by Theorems \ref{thm:interiorApproximation} and \ref{thm:mollification} b) there is $\varepsilon > 0$ small enough such that $v_\varepsilon = \eta_\varepsilon \ast v \in \CC_0^\infty(V_0)$ and 
  $$
  \|v - v_\varepsilon\|_{\WW^{k,p}(\R^n)} = \|v - v_\varepsilon\|_{\WW^{k,p}(V_0)} < \frac{\delta}{2}.
  $$
  Hence, $\|u - v_\varepsilon\|_{\WW^{k,p}(\R^n)} < \delta$ and $v_\varepsilon \in \CC_0^\infty(\R^n)$.
\end{proof}

